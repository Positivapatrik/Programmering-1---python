\section{Flödesdiagram}
\label{section:flowchart}

Flödesdiagram är ett visuellt verktyg för att beskriva algoritmer och programflöden. De hjälper till att strukturera och förstå hur ett program är uppbyggt och fungerar.

\begrepp{Flödesdiagram}{
Ett flödesdiagram är en grafisk representation av en process eller algoritm. Det använder symboler som representerar olika steg och pilar för att visa flödet mellan dessa steg.
}

Symboler i flödesdiagram
I flödesdiagram används följande standardiserade symboler:\\

- **Oval (start/stop):** Representerar början eller slutet på ett flöde.\\
- **Rektangel (process):** Representerar ett steg i processen, till exempel en beräkning.\\
- **Parallelltrapets (in-/utdata):** Representerar inmatning eller utmatning, t.ex. att läsa in eller skriva ut data.\\
- **Romb (beslut):** Representerar ett beslut, t.ex. ett val baserat på en if-sats.\\

Exempel: Enkel sekvens\\
\textbf{Flödesdiagram:}
\begin{center}
\begin{tikzpicture}[node distance=2cm]

\node (start) [startstop] {Start};
\node (input) [io, below of=start] {Läs tal};
\node (process) [process, below of=input] {Addera 5};
\node (output) [io, below of=process] {Skriv resultat};
\node (stop) [startstop, below of=output] {Stop};

\draw [arrow] (start) -- (input);
\draw [arrow] (input) -- (process);
\draw [arrow] (process) -- (output);
\draw [arrow] (output) -- (stop);

\end{tikzpicture}
\end{center}

För att följa ett flödesdiagram så börjar vi i start-rutan, och följer pilarna. 
Vid varje ny symbol så utför vi det som står där, exempelvis fråga användaren efter ett tal. 
Så småningom så kommer vi nå stop-symbolen och då är programmet slut. 
Flödesdiagram är till för att beskriva ett program, men behöver inte skrivas på ett
lika exakt sätt som pythonkod. 
Vi kan därför skriva saker som ''Läs tal'' och räkna med att den som läser
flödesdiagrammet förstår att när vi sen skriver ''Addera 5'' menar vi ''Addera 5 till talet och spara det nya värdet till tal''.
Hur formellt och noga man ska skriva sina flödesdiagram är delvis en smaksak. 
Målet är däremot att det ska vara tydligt hur det fungerar. Kanske tycker du inte det är självklart vad ''Addera 5'' innebär i flödesdiagrammet ovan. 

Vi kan också översätta vårat flödesdiagram till pythonkod:

\textbf{Motsvarande Python-kod:}
\begin{lstlisting}[language=Python]
# Läs ett tal från användaren
nummer = int(input("Ange ett tal: "))

# Lägg till 5
resultat = nummer + 5

# Skriv ut resultatet
print("Resultatet är:", resultat)
\end{lstlisting}

\pythonoutput{Exempel output}{
Ange ett tal: 10\\
Resultatet är: 15
}

Exempel: Beslutsstruktur
I många program måste beslut tas beroende på data. Beslutsstrukturer visas i flödesdiagram med en romb.

\textbf{Flödesdiagram:}
\begin{center}
\begin{tikzpicture}[node distance=2cm]

\node (start) [startstop] {Start};
\node (input) [io, below of=start] {Läs tal};
\node (decision) [decision, below of=input] {Är talet positivt?};
\node (process) [process, right of=decision, xshift=4cm] {Addera 5};
\node (output1) [io, below of=process] {Skriv resultat};
\node (output2) [io, below of=decision, yshift=-1cm] {Skriv Negativt};
\node (stop) [startstop, below of=output2] {Stop};

\draw [arrow] (start) -- (input);
\draw [arrow] (input) -- (decision);
\draw [arrow] (decision.east) -- node[anchor=south] {Ja} (process.west);
\draw [arrow] (process) -- (output1);
\draw [arrow] (decision.south) -- node[anchor=east] {Nej} (output2.north);
\draw [arrow] (output1.south) |- (stop.east);
\draw [arrow] (output2) -- (stop);

\end{tikzpicture}
\end{center}

I programmet ser du romb eller diamantsymbolen som motsvarar en if-sats i programmeringskod. 
Det är den ända symbolen i flödesdiagram som det kan gå ut mer än en pil från. 
Notera också att pilarna som går ut från diamantsymbolen är markerade med Ja och Nej för att visa
vid vilka fall vi ska gå vilken väg. 
I detta fall så går vi på Ja-pilen om talet är positivt. Eller Nej pilen om talet inte är positivt. 
Det som står i en diamantsymbol är alltså alltid en ja-nej-fråga. 

\textbf{Motsvarande Python-kod:}
\begin{lstlisting}[language=Python]
# Läs ett tal från användaren
nummer = int(input("Ange ett tal: "))

if nummer > 0:
    # Om talet är positivt, addera 5
    resultat = nummer + 5
    print("Resultatet är:", resultat)
else:
    # Om talet är negativt, skriv "Negativt"
    print("Talet är negativt")
\end{lstlisting}

\pythonoutput{Exempel output 1}{
Ange ett tal: 3\\
Resultatet är: 8
}

\pythonoutput{Exempel output 2}{
Ange ett tal: -5\\
Talet är negativt
}

Om vi har ett program med en while-loop är det ända vi behöver göra
att dra en pil tillbaka till ett tidigare steg. 
En while-loop avbryts när ett villkor är falskt så vi kan använda
våran vanliga diamantsymbol. 
Genom att på detta sätt återanvända delar av ett flödesdiagram kan vi på samma
sätt som i programmeringskod beskriva mera komplicerade processer.

\begin{tabular*}{\linewidth}{@{\extracolsep{\fill}} p{0.55\linewidth} | p{0.45\linewidth}}
\textbf{Flödesdiagram:} & \textbf{Python-kod:} \\
\hline

% Flödesdiagram
\raisebox{-\totalheight}{%
\begin{tikzpicture}[node distance=2cm]
\node (start) [startstop] {Start};
\node (summainit) [process, below of=start] {Sätt summa till 0};
\node (talinit) [process, below of=summainit] {Sätt tal till 0};
\node (decision) [decision, below of=talinit] {summa < 1000?};
\node (summera) [process, below of=decision] {summa <- summa + tal};
\node (ökatal) [process, below of=summera] {Öka tal med 1};
\node (stop) [startstop, right of=decision, xshift=2.5cm] {Stop};

\draw [arrow] (start) -- (summainit);
\draw [arrow] (summainit) -- (talinit);
\draw [arrow] (talinit) -- (decision);
\draw [arrow] (decision.east) -- node[anchor=south] {ja} (stop.west);
\draw [arrow] (decision.south) -- node[anchor=east] {nej} (summera);
\draw [arrow] (summera) -- (ökatal);
\draw [arrow] (ökatal.west) -- ++(-1,0) -- ++(0,4) -- (decision.west);

%\draw [arrow] (ökatal) -- (decision.west);

\end{tikzpicture}
}
&

% Python-kod
\raggedright
\begin{lstlisting}[caption={Summera positiva tal}, xleftmargin=1.5em]
# Räknar 1+2+3+4+5... så länge summan är mindre än 1000
summa = 0
tal = 1
while summa < 1000:
    summa = summa + tal

print("Summa 1+2+3+...+",tal,">1000")

\end{lstlisting}
\vspace{0.5em}
Räknar ut vilket tal vi behöver räkna till så att $1+2+3+\cdots+tal>1000$.
\\
\end{tabular*}
\observera{I detta flödesdiagram har vi varit lite mer noga med att beskriva variabler än i tidigare exempel. 
Det är som sagt inte en exakt vetenskap hur det ska beskrivas, men eftersom vi i programmet håller reda på både 
summan av alla tal, och vilket tal vi ligger på, kan det vara bra att vara lite extra tydlig. 
Notera också att vi skrivit \textbf{summa <- summa + tal} istället för \textbf{summa = summa + tal}. 
Det är vanligt att man använder en pil för att beskriva tilldelning till en variabel när man skriver flödesdiagram, eller psuedokod vilket vi tar upp i ett senare avsnitt
}


Sammanfattning:
Flödesdiagram är ett kraftfullt verktyg för att planera och analysera programlogik. Genom att översätta diagram till kod (och vice versa) kan du få en bättre förståelse för hur program fungerar.

\secToExRef{examples:flowchart}

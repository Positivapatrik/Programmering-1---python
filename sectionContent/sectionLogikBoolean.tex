\section{Logiska uttryck och booleanska värden}
\label{section:boolean}
\subsection{Vad är ett logiskt uttryck?}
Ett logiskt uttryck är ett uttryck som antingen är \textbf{sant} (\texttt{True}) eller \textbf{falskt} (\texttt{False}). Logiska uttryck används ofta i beslutsfattande, till exempel i \texttt{if}-satser och loopar, för att avgöra om en viss kod ska köras.

\begin{lstlisting}[title=Exempel på logiska uttryck]
x = 5
y = 10
print(x < y)  # True
print(x == y) # False
\end{lstlisting}

I exemplet ovan är \texttt{x < y} sant eftersom \texttt{5} är mindre än \texttt{10}. Däremot är \texttt{x == y} falskt eftersom \texttt{5} inte är lika med \texttt{10}.

\begrepp{Boolean}{En datatyp i Python som kan vara \texttt{True} (sant) eller \texttt{False} (falskt). Den används i logiska uttryck.}

\subsection{Logiska operatorer}
Python har tre logiska operatorer: \texttt{and}, \texttt{or}, och \texttt{not}. Dessa används för att kombinera eller invertera logiska uttryck.

\subsubsection{\textbf{and}}
Operatorn \texttt{and} returnerar \texttt{True} om båda uttrycken är sanna.
\begin{lstlisting}[title=Exempel med \texttt{and}]
x = 5
y = 10
z = 15

print(x < y and y < z)  # True (båda villkoren är sanna)
print(x > y and y < z)  # False (det första villkoret är falskt)
\end{lstlisting}

\subsubsection{\textbf{or}}
Operatorn \texttt{or} returnerar \texttt{True} om minst ett av uttrycken är sant.
\begin{lstlisting}[title=Exempel med \texttt{or}]
x = 5
y = 10

print(x > y or x < y)   # True (det andra villkoret är sant)
print(x > y or y > 20)  # False (båda villkoren är falska)
\end{lstlisting}

\subsubsection{\textbf{not}}
Operatorn \texttt{not} inverterar värdet av ett logiskt uttryck.
\begin{lstlisting}[title=Exempel med \texttt{not}]
x = 5
y = 10

print(not x < y)  # False (inverterar värdet av "x < y", som är True)
\end{lstlisting}

\subsection{Praktiskt exempel: Kontrollera behörighet}
Här är ett exempel där logiska operatorer används för att avgöra om en person är behörig att rösta.
\begin{lstlisting}[title=Exempel med behörighetskontroll]
alder = int(input("Hur gammal är du? "))

if alder >= 18 and alder < 120:
    print("Du får rösta!")
else:
    print("Du är inte behörig att rösta.")
\end{lstlisting}

I detta program kontrolleras om åldern är 18 år eller äldre och mindre än 120.

\subsection{Övningar}
\begin{exercise}
Skriv ett program som frågar användaren efter ett nummer. Programmet ska skriva ut \texttt{True} om numret är mellan 10 och 20, annars \texttt{False}.
\end{exercise}

\begin{solution}
\begin{lstlisting}
nummer = int(input("Skriv in ett nummer: "))
print(nummer >= 10 and nummer <= 20)
\end{lstlisting}
\end{solution}

\begin{exercise}
Skriv ett program som frågar användaren efter två tal. Programmet ska skriva ut \texttt{True} om minst ett av talen är större än 50.
\end{exercise}

\begin{solution}
\begin{lstlisting}
tal1 = int(input("Skriv in det första talet: "))
tal2 = int(input("Skriv in det andra talet: "))
print(tal1 > 50 or tal2 > 50)
\end{lstlisting}
\end{solution}

\begin{exercise}
Skriv ett program som frågar användaren efter ett lösenord. Om lösenordet är \texttt{hemligt123} ska programmet skriva \texttt{Access granted}, annars \texttt{Access denied}.
\end{exercise}

\begin{solution}
\begin{lstlisting}
losenord = input("Skriv in ditt lösenord: ")

if losenord == "hemligt123":
    print("Access granted")
else:
    print("Access denied")
\end{lstlisting}
\end{solution}

\secToExRef{examples:boolean}

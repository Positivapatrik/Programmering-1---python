\section{Pseudokod}
\label{section:psuedocode}
Pseudokod är en metod för att beskriva algoritmer på ett sätt som är lätt att läsa och förstå, utan att behöva använda den exakta syntaxen i ett programmeringsspråk. Det fungerar som en mellanliggande representation som hjälper oss att planera och strukturera vår kod innan vi implementerar den.

\subsection{Varför använda pseudokod?}
Pseudokod är användbart eftersom det:
\begin{itemize}
    \item Hjälper till att fokusera på logiken i en algoritm, utan att fastna i språkspecifik syntax.
    \item Är enkelt att läsa och förstå, även för personer som inte är programmerare.
    \item Underlättar planeringen av mer komplexa program.
\end{itemize}

\subsection{Hur skriver man pseudokod?}
Det finns inga fasta regler för pseudokod, men här är några riktlinjer:
\begin{itemize}
    \item Använd beskrivande namn för variabler och steg.
    \item Håll det kortfattat och tydligt.
    \item Strukturera koden med indragningar för att visa block som hör ihop.
    \item Använd enkla termer som "LOOP", "IF", och "ELSE".
\end{itemize}

\subsection{Exempel: Summera en lista}
Låt oss skriva pseudokod för att summera alla värden i en lista.

\textbf{Pseudokod:}
\begin{verbatim}
START
SET summa TO 0
FOR varje element i listan:
    ADD element TO summa
END FOR
PRINT summa
END
\end{verbatim}

\textbf{Python-implementation:}
\begin{lstlisting}[title=Summera en lista]
lista = [1, 2, 3, 4, 5]
summa = 0
for element in lista:
    summa += element
print(summa)  # Output: 15
\end{lstlisting}

\subsection{Exempel: Hitta det största talet i en lista}
Här är ett exempel på pseudokod för att hitta det största talet i en lista.

\textbf{Pseudokod:}
\begin{verbatim}
START
SET största TO första elementet i listan
FOR varje element i listan:
    IF element > största:
        SET största TO element
    END IF
END FOR
PRINT största
END
\end{verbatim}

\textbf{Python-implementation:}
\begin{lstlisting}[title=Hitta största talet i en lista]
lista = [10, 20, 5, 30, 15]
största = lista[0]
for element in lista:
    if element > största:
        största = element
print(största)  # Output: 30
\end{lstlisting}

\subsection{Övningar}
\begin{exercise}
Skriv pseudokod för att räkna hur många jämna tal som finns i en lista.
\end{exercise}

\begin{exercise}
Implementera följande pseudokod i Python: \\
\texttt{START \\
SET antal TO 0 \\
FOR varje element i listan: \\
    IF element är större än 10: \\
        ADD 1 TO antal \\
    END IF \\
END FOR \\
PRINT antal \\
END}
\end{exercise}

\begin{rconceptbox}{Pseudokod}{Ett sätt att beskriva algoritmer med ord och strukturer som påminner om kod, men utan att följa strikt syntax.}
\end{rconceptbox}

\secToExRef{examples:psuedocode}

\section{Funktioner: Strukturera och Återanvänd Kod}
\label{section:functions}

Funktioner är en grundläggande del av programmering. De låter oss strukturera kod genom att dela upp den i mindre, hanterbara delar. Funktioner kan också återanvändas, vilket gör vårt arbete effektivare och vår kod lättare att förstå och underhålla.

\subsection{Vad är en funktion?}
En funktion är en bit kod som utför en specifik uppgift. 
Vi definierar funktioner med nyckelordet \texttt{def} och ett namn, följt av eventuella parametrar inom parentes. 
Funktionen kan köras genom att vi anropar dess namn.

\begin{lstlisting}[title=En enkel funktion]
def hälsa():
    print("Hej! Välkommen till Python.")
    
hälsa()  # Anropa funktionen
\end{lstlisting}

\subsection{Parametrar och argument}
Vi kan ge funktioner indata genom att använda parametrar. När vi anropar funktionen skickar vi värden, som kallas argument, till parametrarna.

\begin{lstlisting}[title=Funktion med parametrar]
def hälsa(namn):
    print(f"Hej, {namn}! Välkommen till Python.")
    
hälsa("Alice")  # Output: Hej, Alice! Välkommen till Python.
\end{lstlisting}

\begrepp{Parameter}{En parameter är en plats för indata i en funktions definition.}
\begrepp{Argument}{Ett argument är det faktiska värde vi skickar till en funktions parameter.}

\subsection{Returvärden}
En funktion kan också ge tillbaka ett värde till det ställe där den anropas. Detta görs med nyckelordet \texttt{return}.

\begin{lstlisting}[title=Funktion med returvärde]
def addera(a, b):
    return a + b
    
resultat = addera(5, 7)
print(resultat)  # Output: 12
\end{lstlisting}

\begrepp{Returvärde}{Ett returvärde är det värde som en funktion skickar tillbaka till anropsplatsen med hjälp av \texttt{return}.}

\subsection{Main-funktionen}
I större program är det vanligt att använda en \texttt{main}-funktion för att tydligt visa var programmet börjar. Här är ett exempel:

\begin{lstlisting}[title=Ett program med en main-funktion]
def hälsa():
    print("Hej! Välkommen till Python.")
    
def main():
    hälsa()

if __name__ == "__main__":
    main()
\end{lstlisting}

När vi använder \texttt{if \_\_name\_\_ == "\_\_main\_\_"}, ser Python till att bara köra \texttt{main()} om programmet körs direkt, inte om det importeras i ett annat program.

\begrepp{Main-funktion}{En \texttt{main}-funktion fungerar som programmets startpunkt, vilket gör koden tydligare och mer strukturerad.}

\subsection{Namespace: Variabler i funktioner}
Varje funktion har sitt eget \textbf{namespace}, vilket innebär att variabler definierade i funktionen inte påverkar eller är kända utanför den. Här är ett exempel:

\begin{lstlisting}[title=Namespace och variabler]
def ändra_värde():
    x = 10  # Lokal variabel
    print(f"Inne i funktionen: {x}")
    
x = 5  # Global variabel
ändra_värde()
print(f"Utanför funktionen: {x}")
\end{lstlisting}

Output:
\begin{verbatim}
Inne i funktionen: 10
Utanför funktionen: 5
\end{verbatim}

I detta exempel är variabeln \texttt{x} inne i funktionen skild från variabeln \texttt{x} utanför. Detta förhindrar att funktioner oavsiktligt ändrar värden i resten av programmet.

\begrepp{Namespace}{Ett namespace är ett område där variabler och deras värden lagras. En funktions namespace är separat från resten av programmet.}

\subsection{Övningar}
\begin{exercise}
Skriv en funktion \texttt{dubbla()} som tar ett tal som parameter och returnerar dess dubbla värde.
\end{exercise}

\begin{exercise}
Skapa ett program med en \texttt{main}-funktion som anropar en funktion \texttt{hälsa(namn)} och skriver ut en personlig hälsning.
\end{exercise}

\begin{exercise}
Experimentera med namespace genom att skapa en funktion som definierar och ändrar en variabel. Kontrollera om förändringen påverkar en variabel med samma namn utanför funktionen.
\end{exercise}

\secToExRef{examples:functions}

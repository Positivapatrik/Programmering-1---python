\section{For-loop (Upprepa saker med Python)}
\label{section:for}

En \texttt{for}-loop används när vi vill upprepa något för varje sak i en lista, varje bokstav i en text, eller över en rad nummer. Tanken är enkel: loopen går igenom varje sak i listan, en i taget, och gör något med den.

\subsection{Hur fungerar en for-loop?}

En \texttt{for}-loop i Python ser ut så här:

\begin{lstlisting}[title=Syntax för for-loop]
for variabel in lista:
    # Gör något med variabeln
\end{lstlisting}

Vad händer steg för steg?\\

1. \textbf{Variabeln} är ett namn vi väljer. Den kommer att innehålla en sak från listan i taget.\\
2. \textbf{Listan} är det vi vill gå igenom, t.ex. en samling siffror eller bokstäver.\\
3. Loopen tar varje sak i listan och lagrar den i variabeln. Sedan kör den koden inuti loopen.\\

Exempel: Skriv ut siffror i en lista\\

Här är ett enkelt exempel där vi har en lista med siffror:

\begin{lstlisting}[title=Exempel: Skriv ut siffror]
tal_lista = [1, 2, 3, 4, 5]

for tal in tal_lista:
    print(tal)
\end{lstlisting}

Vad händer när programmet körs?

1. Första gången i loopen är \texttt{tal = 1}. Koden \texttt{print(tal)} körs och skriver ut 1.\\
2. Sedan hoppar loopen till nästa tal i listan, \texttt{tal = 2}, och skriver ut 2.\\
3. Detta fortsätter för alla tal i listan.\\

Resultatet blir:

\pythonoutput{Exempel: Output}{
1 \\
2 \\
3 \\
4 \\
5
}

\observera{Det är viktigt att koden inuti loopen är indenterad (indragen), annars får du fel. Python använder indrag för att veta vad som hör till loopen.}

\subsection{Ett tydligare exempel: Steg för steg}

Låt oss se ett ännu tydligare exempel. Här använder vi en lista med namn:

\begin{lstlisting}[title=Exempel: Skriv ut namn]
namn_lista = ["Anna", "Björn", "Cecilia"]

for namn in namn_lista:
    print("Hej, " + namn + "!")
\end{lstlisting}

Vad händer när programmet körs?

1. \textbf{Första varvet}: Variabeln \texttt{namn} blir \texttt{''Anna''}. Programmet skriver ut \texttt{"Hej, Anna!"}.\\
2. \textbf{Andra varvet}: Variabeln \texttt{namn} blir \texttt{''Björn''}. Programmet skriver ut \texttt{"Hej, Björn!"}.\\
3. \textbf{Tredje varvet}: Variabeln \texttt{namn} blir \texttt{''Cecilia''}. Programmet skriver ut \texttt{"Hej, Cecilia!"}.\\

Outputen blir:

\pythonoutput{Exempel: Output}{
Hej, Anna! \\
Hej, Björn! \\
Hej, Cecilia!
}


\subsection{For-loop med \texttt{range()}}
Vi kan också använda \texttt{range()} för att skapa en serie siffror automatiskt. Till exempel kan vi skriva ut talen från 1 till 5 utan att skapa en lista först:

\begin{lstlisting}[title=Exempel: range()]
for tal in range(1, 6):
    print(tal)
\end{lstlisting}

Här skapar \texttt{range(1, 6)} siffrorna 1 till 5 (men inte 6). Resultatet blir detsamma som med listan ovan.

\textbf{Steg i \texttt{range()}}\\
Du kan lägga till ett steg i \texttt{range()} för att hoppa över tal. Till exempel:
\begin{lstlisting}
for tal in range(2, 11, 2):
    print(tal)
\end{lstlisting}
Här hoppar \texttt{range()} två steg åt gången och skriver ut 2, 4, 6, 8, 10.


\subsection{For-loop med text}
En sträng i Python kan behandlas som en lista av bokstäver. Vi kan använda en \texttt{for}-loop för att skriva ut varje bokstav i en sträng:

\begin{lstlisting}[title=Exempel: Loopa över en sträng]
text = "Python"

for bokstav in text:
    print(bokstav)
\end{lstlisting}

Outputen blir:
\pythonoutput{Exempel: Output}{
P \\
y \\
t \\
h \\
o \\
n
}



\subsection{Övningar}
\begin{exercise}
Skriv ett program som skriver ut alla siffror från 1 till 10, en per rad.
\end{exercise}
\begin{solution}
\begin{lstlisting}
for tal in range(1, 11):
    print(tal)
\end{lstlisting}
\end{solution}

\begin{exercise}
Skriv ett program som skriver ut alla multiplar av 3 mellan 1 och 30.
\end{exercise}
\begin{solution}
\begin{lstlisting}
for tal in range(3, 31, 3):
    print(tal)
\end{lstlisting}
\end{solution}

\begin{exercise}
Skriv ett program som skriver ut varje bokstav i strängen ''Programmering'' på en ny rad.
\end{exercise}
\begin{solution}
\begin{lstlisting}
text = "Programmering"
for bokstav in text:
    print(bokstav)
\end{lstlisting}
\end{solution}

\secToExRef{examples:for}

\section{Listor i Python}
\label{section:lists}
Listor är en av de mest användbara datatyperna i Python. De används för att lagra flera värden i en och samma variabel. Till exempel kan en lista innehålla en samling av tal, strängar eller till och med andra listor.

\subsection{Skapa en lista}
En lista skapas genom att använda hakparenteser \texttt{[]} och separera elementen med kommatecken:
\begin{lstlisting}[title=Exempel: Skapa en lista]
# En lista med tal
tal = [1, 2, 3, 4, 5]

# En lista med text
djur = ["katt", "hund", "kanin"]

# En blandad lista
blandad = [1, "äpple", True]
\end{lstlisting}

\begrepp{Lista}{En lista är en samling av värden som kan lagras i en variabel.}
\begrepp{Element}{Ett värde i listan kallas för ett element}

\subsection{Åtkomst till element i en lista}
Varje element i en lista har ett index som börjar på \texttt{0} för det första elementet. Du kan komma åt ett element genom att ange dess index inom hakparenteser:
\begin{lstlisting}[title=Exempel: Åtkomst via index]
djur = ["katt", "hund", "kanin"]
print(djur[0])  # Skriv ut "katt"
print(djur[1])  # Skriv ut "hund"
\end{lstlisting}

\begrepp{Index}{Index är platsen på ett av listans värden. I python anges första elementet med index 0.}

\subsection{Modifiera en lista}
Du kan ändra värdet på ett element, lägga till nya element eller ta bort befintliga element.

\paragraph{Ändra värde:}
\begin{lstlisting}[title=Exempel: Ändra värde]
djur = ["katt", "hund", "kanin"]
djur[1] = "hamster"  # Ändra "hund" till "hamster"
print(djur)  # ["katt", "hamster", "kanin"]
\end{lstlisting}

\paragraph{Lägga till element:}
\begin{lstlisting}[title=Exempel: Lägga till element]
djur = ["katt", "hund"]
djur.append("kanin")  # Lägg till "kanin" sist i listan
print(djur)  # ["katt", "hund", "kanin"]
\end{lstlisting}

\paragraph{Ta bort element:}
\begin{lstlisting}[title=Exempel: Ta bort element]
djur = ["katt", "hund", "kanin"]
djur.remove("hund")  # Ta bort "hund"
print(djur)  # ["katt", "kanin"]
\end{lstlisting}

\subsection{Loopar genom en lista}
Det är vanligt att använda loopar för att bearbeta alla element i en lista.
Vi kommer i nästa avsnitt se att man vanligen gör detta med en for-loop. 
Vi använder i detta exempel en while-loop eftersom det är vad vi sett hittills. 
\begin{lstlisting}[title=Exempel: For-loop med en lista]
djur = ["katt", "hund", "kanin"]
i = 0
while i<djur.len(): #Kör loopen tills i når slutet på listan
  print(djur[i])
  i = i + 1

# Skriver ut:
# katt
# hund
# kanin
\end{lstlisting}

Notera att vi kör programmet till variabeln \textbf{i} har ett värde som är ett mindre än listans längd.
Eftersom vi börjar räkna på 0 får vi ändå med alla tal. 
Inuti loopen använder vi våran variabel \textbf{i} för att skriva ut
ett element i listan i taget. 

\observera{Vi använder vanligen en variabel med namnet \textbf{i} för att beskriva ett index}

\subsection{Index out of bounds}
Ett vanligt fel när man jobbar med listor är ett ''Index out of bounds''-error.
Det betyder att vi har försökt hämta ett element (Ett värde i listan) som ligger utanför listan.
Om vi i en lista med 4 element försöker hämta det 5 elementet får vi det felet.
\begin{lstlisting}[title=Index out of bounds error]
tal = [1,2,3,4]
print(tal[5])
\end{lstlisting}

\begrepp{Index out of bounds-error}{En typ av fel vi får när vi försöker hämta ett element från en lista som ligger utanför listan.}

\subsection{Vanliga metoder för listor}
Python erbjuder flera metoder för att arbeta med listor:
\begin{itemize}
    \item \texttt{append()}: Lägg till ett element i slutet.
    \item \texttt{remove()}: Ta bort ett specifikt element.
    \item \texttt{pop()}: Ta bort och returnera det sista elementet (eller ett specifikt index).
    \item \texttt{sort()}: Sortera listan.
    \item \texttt{len()}: Returnera antalet element i listan.
\end{itemize}
\begin{lstlisting}[title=Exempel: Använda lista-metoder]
djur = ["kanin", "hund", "katt"]
djur.sort()  # Sortera listan
print(djur)  # ["hund", "kanin", "katt"]
\end{lstlisting}

\subsection{Övningar}
\begin{exercise}
Skapa en lista med namnen på tre frukter och skriv ut dem en och en med en loop.
\end{exercise}
\begin{solution}
\begin{lstlisting}
frukter = ["äpple", "banan", "apelsin"]
for frukt in frukter:
    print(frukt)
\end{lstlisting}
\end{solution}

\begin{exercise}
Skapa en lista med siffrorna 1 till 5. Lägg till talet 6 och skriv sedan ut hela listan.
\end{exercise}
\begin{solution}
\begin{lstlisting}
tal = [1, 2, 3, 4, 5]
tal.append(6)
print(tal)  # [1, 2, 3, 4, 5, 6]
\end{lstlisting}
\end{solution}

\begin{exercise}
Skapa en lista med talen 10, 20 och 30. Använd \texttt{len()} för att skriva ut antalet element i listan.
\end{exercise}
\begin{solution}
\begin{lstlisting}
tal = [10, 20, 30]
print(len(tal))  # 3
\end{lstlisting}
\end{solution}

\secToExRef{examples:lists}


\chapter{Introduktion}
\label{chapter:intro}
Välkommen till denna bok om programmering med \textbf{Python}. Här lär du dig de grundläggande byggstenarna för att skriva och förstå kod. Vi går steg för steg med enkla exempel och övningar. 

Kom ihåg att stanna upp och läs igen om du inte är säker på om du förstått. 
Övningarna syftar till att kontrollera att man också själv kan använda det som avsnitten förklarar. 
I slutet på boken finns lösningsförslag på övningarna. 
I programmering kan det finnas många olika sätt att lösa en uppgift på. 
Lösningsförslagen är inte nödvändigtviss det ända eller ens det bästa sätten, men ett förslag på hur man kan tänka. 

\section{Kodexempel}
Förutom kodexempel i varje avsnitt finns också ett appendix med fler kodexempel.
Det kan vara bra att kolla på fler exempel för att se en större variation av användning av de programmeringskoncept vi kollar på i avsnitten.

\section{Vad är programmering?}
Programmering handlar om att skriva instruktioner som en dator kan förstå och följa. Python är ett av de mest populära programmeringsspråken eftersom det är lätt att läsa och använda. Det är också ett vanligt val inom maskininlärning, eller webbutveckling. 

\begrepp{Python}{Python är ett programmeringsspråk. Det används för att skapa allt från enkla program till avancerade system som spel och webbplatser.}


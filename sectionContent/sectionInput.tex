

\section{Input (Att ta emot data från användaren)}
\label{section:input}

\subsection{Vad är input?}
I Python används funktionen \texttt{input()} för att ta emot data från användaren. Med \texttt{input()} kan program bli interaktiva genom att ställa frågor eller vänta på användarens svar.

\begin{lstlisting}[title=Exempel på input]
name = input("Vad heter du? ")
print("Hej, " + name + "!")
\end{lstlisting}

I det här exemplet:
- Användaren blir tillfrågad \texttt{"Vad heter du?"}.
- Svaret lagras i variabeln \texttt{name}.
- Programmet använder \texttt{print()} för att hälsa på användaren med det inskrivna namnet.

\begrepp{Input}{Data som användaren skickar in i programmet, ofta via tangentbordet.}


\subsection{Arbeta med olika datatyper}
Funktionen \texttt{input()} returnerar alltid text (strängar). Om du vill använda det inskrivna värdet som ett tal måste du omvandla det med \texttt{int()} (heltal) eller \texttt{float()} (decimaltal).

\begin{lstlisting}[title=Exempel på input med tal]
age = int(input("Hur gammal är du? "))
age = age + 1
print("Nästa år fyller du " + str(age) + " år.")
\end{lstlisting}

I detta exempel:\\
1. \texttt{input()} tar emot en text. Det vill säga en string. \\
2. \texttt{int()} konverterar string-värdet till ett heltal. 
Eftersom vi omvandlat våran variabel till en int kan vi sen göra en beräkning med den \textbf{age=age+1}.\\
3. \texttt{str()} används för att omvandla talet tillbaka till text i utskriften.\\

\subsection{Övningar}
\begin{exercise}
Skriv ett program som frågar användaren efter deras favoritfärg och sedan skriver ut \texttt{"Din favoritfärg är [färgen]"}. 
\end{exercise}

\begin{solution}
\begin{lstlisting}
color = input("Vad är din favoritfärg? ")
print("Din favoritfärg är " + color + ".")
\end{lstlisting}
\end{solution}

\begin{exercise}
Skapa ett program som frågar användaren efter två tal och sedan skriver ut summan av dem.
\end{exercise}

\begin{solution}
\begin{lstlisting}
num1 = int(input("Ange det första talet: "))
num2 = int(input("Ange det andra talet: "))
print("Summan är: " + str(num1 + num2))
\end{lstlisting}
\end{solution}

\begin{exercise}
Skriv ett program som frågar användaren efter deras ålder. Om åldern är mindre än 18, skriv ut \texttt{"Du är inte myndig."}. Annars, skriv ut \texttt{"Du är myndig."}.
\end{exercise}

\begin{solution}
\begin{lstlisting}
age = int(input("Hur gammal är du? "))
if age < 18:
    print("Du är inte myndig.")
else:
    print("Du är myndig.")
\end{lstlisting}
\end{solution}

\subsection{Sammanfattning}
- Funktionen \texttt{input()} används för att ta emot data från användaren.  \\
- Data från \texttt{input()} är alltid text (strängar). Omvandling behövs för att arbeta med tal.  \\

\secToExRef{examples:input}

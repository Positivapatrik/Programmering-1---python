\section{Bubble Sort}
\label{section:bubblesort}

\subsection*{Vad är Bubble Sort?}
Bubble Sort är en enkel sorteringsalgoritm som fungerar genom att jämföra två intilliggande element i en lista och byta plats på dem om de är i fel ordning. Den upprepar detta tills hela listan är sorterad. Namnet "Bubble Sort" kommer från att det största (eller minsta) elementet "bubblar" upp till sin rätta plats i varje iteration.

\subsection*{Hur fungerar Bubble Sort?}
Algoritmen går igenom listan flera gånger:
\begin{itemize}
    \item Jämför två intilliggande element.
    \item Om de är i fel ordning, byt plats på dem.
    \item Fortsätt jämföra nästa par tills slutet av listan är nådd.
    \item Upprepa detta tills hela listan är sorterad.
\end{itemize}

\subsection*{Kodexempel}
Här är en implementation av Bubble Sort i Python:
\begin{lstlisting}[title=Bubble Sort i Python]
def bubble_sort(lista):
    n = len(lista)
    for i in range(n):
        for j in range(0, n-i-1):
            if lista[j] > lista[j+1]:
                # Byt plats
                lista[j], lista[j+1] = lista[j+1], lista[j]

# Exempel
data = [64, 34, 25, 12, 22, 11, 90]
bubble_sort(data)
print("Sorterad lista:", data)
\end{lstlisting}

\subsection*{Förklaring av koden}
\begin{enumerate}
    \item \texttt{for i in range(n)}: Loopar genom listan flera gånger.
    \item \texttt{for j in range(0, n-i-1)}: Ser till att algoritmen inte jämför element som redan är sorterade.
    \item \texttt{if lista[j] > lista[j+1]}: Kontrollerar om elementen är i fel ordning.
    \item \texttt{lista[j], lista[j+1] = lista[j+1], lista[j]}: Byter plats på elementen.
\end{enumerate}

\subsection*{Visualisering}
Bubble Sort kan visualiseras som att sorteringen sker steg för steg, där varje större element "flyter upp" till toppen.

\subsection*{Övning}
\begin{exercise}
Implementera Bubble Sort för en lista som innehåller följande värden: \texttt{[10, 8, 2, 7, 1, 3]} och skriv ut den sorterade listan.
\end{exercise}
\begin{solution}
\begin{lstlisting}
data = [10, 8, 2, 7, 1, 3]
bubble_sort(data)
print("Sorterad lista:", data)
\end{lstlisting}
Output: \texttt{[1, 2, 3, 7, 8, 10]}
\end{solution}

\subsection*{Nackdelar med Bubble Sort}
Bubble Sort är enkel att förstå, men den är inte särskilt effektiv för långa listor eftersom den har en tidskomplexitet på \(O(n^2)\). Detta innebär att tiden det tar att sortera listan ökar kvadratiskt med dess storlek. För större datamängder är mer avancerade algoritmer, som Quick Sort eller Merge Sort, bättre val.

\secToExRef{examples:bubblesort}

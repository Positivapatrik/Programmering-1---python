
\section{Att använda slumptal och modulen \texttt{random}}
\label{section:random}
\subsection{Vad är en modul?}
En modul i Python är en samling fördefinierad kod som vi kan använda i våra program. Moduler är som bibliotek som innehåller användbara funktioner och verktyg som kan spara oss tid och arbete. För att använda en modul måste vi först \textbf{importera} den.

\begin{lstlisting}[title=Exempel på import]
import random
\end{lstlisting}

I detta exempel importeras modulen \texttt{random}, som innehåller funktioner för att arbeta med slumpmässighet.

\begrepp{Modul}{En modul är en samling färdiga funktioner och verktyg som vi kan använda i våra program.}

\subsection{Slumptal med \texttt{random.randint()}}
Funktionen \texttt{randint()} från modulen \texttt{random} genererar ett heltal mellan två givna värden. Det är användbart när vi vill simulera slumpmässiga val, som att kasta en tärning.

\begin{lstlisting}[title=Exempel med \texttt{randint()}]
import random

tarning = random.randint(1, 6)
print("Du slog:", tarning)
\end{lstlisting}

I detta program:\\
- \texttt{random.randint(1, 6)} genererar ett heltal mellan 1 och 6 (inklusive båda).\\
- Det slumpmässiga talet lagras i variabeln \texttt{tarning} och skrivs ut med \texttt{print()}.\\

\begrepp{Slumptal}{Ett nummer som genereras slumpmässigt av datorn.}

\observera{randint är en funktion på samma sätt som print.
Funktionen randint finns däremot inte förinladdad när vi startar ett program och vi 
måste därför importera den från random-modulen. 
Vi ser också att vi till en funktion kan skicka in flera argument. I detta fall
skickar vi in två tal som anger mellan vilka tal vi vill få ett slumpat tal. 
När vi skickar flera argument separerar vi dom med kommatecken.}

\subsection{Exempel: Ett enkelt spel}
Här är ett program som använder \texttt{randint()} för att skapa ett enkelt spel där användaren ska gissa ett tal.

\begin{lstlisting}[title=Gissa ett tal-spel]
import random

hemligt_tal = random.randint(1, 10)
gissning = int(input("Gissa ett tal mellan 1 och 10: "))

if gissning == hemligt_tal:
    print("Grattis! Du gissade rätt.")
else:
    print("Fel! Det rätta talet var", hemligt_tal)
\end{lstlisting}

\subsection{Andra användbara funktioner i \texttt{random}}
Modulen \texttt{random} innehåller fler funktioner än bara \texttt{randint}. \\
Här är några exempel:\\
- \texttt{random.random()} genererar ett slumptal mellan 0 och 1.\\
- \texttt{random.choice()} väljer slumpmässigt ett objekt från en lista. 
Vi kommer kolla mer på listor senare. \\

\begin{lstlisting}[title=Exempel med \texttt{random.choice()}]
import random

farger = ["röd", "blå", "grön", "gul"]
vald_farg = random.choice(farger)
print("Den valda färgen är:", vald_farg)
\end{lstlisting}

För att ta reda på vad för funktioner som finns brukar det vara bäst att göra en sökning.
Vanligen får man bättre resultat om man söker på engelska, exempelvis sökning ''python generate random number''.

\subsection{Övningar}
\begin{exercise}
Skriv ett program som slumpar ett tal mellan 1 och 100 och skriver ut det.
\end{exercise}

\begin{solution}
\begin{lstlisting}
import random
tal = random.randint(1, 100)
print("Slumptalet är:", tal)
\end{lstlisting}
\end{solution}

\begin{exercise}
Skapa ett program som kastar två tärningar (med värden mellan 1 och 6) och skriver ut summan av deras resultat.
\end{exercise}

\begin{solution}
\begin{lstlisting}
import random
tarning1 = random.randint(1, 6)
tarning2 = random.randint(1, 6)
print("Summan är:", tarning1 + tarning2)
\end{lstlisting}
\end{solution}

\begin{exercise}
Gör ett program som väljer slumpmässigt en aktivitet från en lista olika aktiviteter, till exempel ''läsa'', ''spela spel'', eller ''träna''.
\end{exercise}

\begin{solution}
\begin{lstlisting}
import random
aktiviteter = ["läsa", "spela spel", "träna", "rita", "baka"]
vald_aktivitet = random.choice(aktiviteter)
print("Idag ska du:", vald_aktivitet)
\end{lstlisting}
\end{solution}

\subsection{Sammanfattning}
- Moduler som \texttt{random} låter oss använda fördefinierade funktioner för specifika ändamål.  \\
- Funktionen \texttt{randint()} används för att skapa slumptal inom ett visst intervall.  \\
- Interaktiva program kan skapas genom att kombinera \texttt{input()} och slumpfunktioner.\\

\secToExRef{examples:random}

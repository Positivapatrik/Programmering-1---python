\section{Variabler och Sekvensiell Exekvering}
\label{section:variables}
\subsection{Vad är en variabel?}
En variabel är som en namngiven låda där vi kan lagra data, som till exempel tal eller text. Vi kan senare ändra innehållet i lådan eller använda det i beräkningar.

\begin{lstlisting}[title=Tilldelning]
minvariabel = 100+1
\end{lstlisting}

I programmet ser vi hur vi sparar ett värde i en variabel. 
I exemplet heter våran variabel \textbf{minvariabel} och vi sparar värdet 101 i den.
Vi kan senare i programmet återanvända våran variabel för att hämta det värde vi sparat i den. 
Notera att vi använder likamedtecknet = men inte på samma sätt som i matematik. 
När python ser = så kommer den först göra en uträkning av det som står på höger sida. 
Det blir alltid ett konkret värde, exempelvis ett tal, eller en text. 
I exemplet räknar python ut åt oss att $100+1$ blir $101$. 
Python sparar sedan värdet $101$ i variabeln som vi döpt till \textbf{minvariabel}.

\begrepp{Likamedtecknet}{Likamedtecknet betyder i programmering att vi sparar värdet på högersidan i variabeln vi skrivit på vänstersidan.}
\observera{Likamedtecken i programmering betyder \textbf{INTE} samma sak som i matematik. Det kan se ut som att vi ställt upp en ekvation, 
men vi använder = bara för att spara ett uträknat värde.}


\subsection{Använda variabler}


\begin{lstlisting}[title=Exempel på variabler]
x = 5
y = 10
z = x + y
print(z)
\end{lstlisting}

När detta program körs, kommer resultatet \texttt{15} att visas på skärmen eftersom \texttt{x} är 5 och \texttt{y} är 10.

\begrepp{Variabel}{En plats i datorns minne där data, som tal eller text, lagras. Vi ger platsen ett namn för att enkelt kunna använda den i programmet.}

\subsection{Datatyper}
Data i Python kan ha olika typer, som bestämmer vad vi kan göra med den. 
Några vanliga typer:\\
- \texttt{int}: Heltal (till exempel \texttt{5})\\
- \texttt{float}: Decimaltal (till exempel \texttt{3.14})\\
- \texttt{str}: Text, även kallad sträng (till exempel \texttt{"Hej!"})\\

\begin{lstlisting}[title=Exempel på olika datatyper]
tal = 5           # int
decimaltal = 3.14 # float
text = "Hej!"     # str
\end{lstlisting}

Eftersom datorn bara ser 1:or och 0:or i minnet behöver den hålla reda på om det vi sparat
är ett heltal eller en text. Vi kommer gå djupar kring datatyper senare, men nu vet du varför vi behöver citattecken när vi 
skriver en text. 
Datatyper är också förklaringen till varför \texttt{print(4/2)} blir 2.0 och inte 2. 
Python gör nämligen om alla divisioner till float (decimaltal).


\begrepp{Datatyp}{En typ av värde, exempelvis heltal (integer), eller textvärden (string)}
\begrepp{Integer}{Variabler av datatypen integer (förkortat int) innehåller heltalsvärden}
\begrepp{Float}{Variabler av datatypen float innehåller decimaltal}
\begrepp{String}{Variabler av datatypen string (förkortat str) innehåller textvärden. På svenska säger vi också att datatypen är en sträng.}

\subsection{Sekvensiell exekvering}
När ett program körs, läses koden rad för rad, uppifrån och ned. 
Detta kallas sekvensiell exekvering. 
Ordningen på kod spelar alltså mycket stor roll!!!
Här är ett exempel:

\begin{lstlisting}[title=Ordning i koden spelar roll]
x = 5
print(x)
x = 7
print(x)
\end{lstlisting}

Resultatet blir först \texttt{5} och sedan \texttt{7}, eftersom värdet på \texttt{x} ändras innan den skrivs ut andra gången.

\begrepp{Exekvering}{
Exekvering betyder att man kör programmeringskod. 
Det vill säga att datorn steg för steg går igenom koden och utför våra instruktioner.}

\begin{minipage}{\linewidth}
\begin{lstlisting}[title=Användning av samma variabel två gånger]
x = 5
print(x)
x = x+2
print(x)
\end{lstlisting}
\end{minipage}

Detta program skriver också ut 5 och sen 7. 
Det beror på att raden x=x+2 betyder att vi först räknar ut högersidan, 
vilket blir 5+2 eftersom x, när programmet ska köra rad 3, har värdet 5. 
Vi sätter sedan detta värde som nytt värde på variabeln x.
Det går med andra ord bra att spara ett nytt värde i en variabel som beror på det gamla värdet.
Det är också ganska vanligt att man vill göra just det, exempelvis för att öka en variabels värde med Ett, 
vilket vi alltså kan åstadkomma genom \texttt{x=x+1}.

\subsection{Strängkonkatenering}
Om vi har sparat textvärden, det vill säga strängar, i våra variabler kan vi sätta ihop texterna med + tecknet.
Detta kallas konkatenering.

\begin{lstlisting}[title=Konkatenering av strängar]
namn = "Kalle"
efternamn = "Anka"
hela = namn + efternamn
print(hela)
\end{lstlisting}
\pythonoutput{Output}{KalleAnka}

I programmet ser vi att + tecknet används också för texter för att konkatenera eller slå ihop dom. 
Vi sparar resultatet i en ny variabel vi döpt till hela, som vi sedan skriver ut med \texttt{print(hela)}.
Vi hade också kunnat addera textsträngarna direkt med \texttt{print("Kalle"+\string"Anka")}.

\begrepp{Konkatenering}{Konkatenering av texter betyder att vi sätter ihop texterna direkt efter varandra. Konkatenering av texten \textbf{hej} och texten \textbf{då} blir \textbf{hejdå}.}

\subsection{Övningar}
\begin{exercise}
Tilldela värdet 7 till en variabel \textbf{a} och värdet 5 till en variabel med namnet \textbf{b}.
Skriv sen ut summan av dem.
\end{exercise}

\begin{solution}
\begin{lstlisting}
a = 7
b = 3
print(a + b)
\end{lstlisting}
\end{solution}

\begin{exercise}
Deklarera en variabel \texttt{x} och tilldela den värdet \texttt{10}. Ändra sedan värdet på \texttt{x} till \texttt{20} och skriv ut det nya värdet.
\end{exercise}

\begin{solution}
\begin{lstlisting}
x = 10
x = 20
print(x)
\end{lstlisting}
\end{solution}

\begin{exercise}
Skapa en variabel med ditt namn och skriv ut följande med hjälp av variabeln: \texttt{Hej, [ditt namn]!}
\end{exercise}

\begin{solution}
\begin{lstlisting}
namn = "Patrik"
print("Hej, " + namn + "!")
\end{lstlisting}
\end{solution}

\secToExRef{examples:variables}


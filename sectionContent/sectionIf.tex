
\section{If-satser (Beslutsfattande)}
\label{section:if}

\subsection{Vad är en \texttt{if}-sats?}
I programmering används \texttt{if}-satser för att fatta beslut baserat på vissa villkor. En \texttt{if}-sats gör att programmet endast kör en viss del av koden om ett villkor är sant.

\begin{lstlisting}[title=Exempel på if-sats]
x = 10
if x > 5:
    print("x är större än 5")
\end{lstlisting}

I detta exempel kontrollerar programmet om värdet på \texttt{x} är större än 5. Om villkoret är sant (vilket det är, eftersom \texttt{x = 10}), skriver programmet ut \texttt{x är större än 5}.
\begrepp{Villkor}{Ett villkor är ett påstående som är antingen sant eller falskt, till exempel \texttt{x > 5}.}

\observera{En if-sats har alltid ett villkor följt av tecknet \textbf{:}\\
Därefter kommer en eller flera rader som är högerjusterade (se indentering nedan).}

\subsection{Indentering i koden}
Python använder indragning eller högerjusterad kod för att markera vilken kod som hör till en \texttt{if}-sats. All kod som är indragen efter en \texttt{if}-sats körs om villkoret är sant.
Detta kallas för indentering och markerar alltså vilken kod som tillhör vilket kodblock.

\begin{lstlisting}[title=Indragning är viktigt]
x = 10
if x > 5:
    print("Detta är indraget")
    print("Det körs om villkoret är sant")
\end{lstlisting}

I programmet är båda print indenterade vilket gör att dom bara körs när villkoret
\textbf{x är större än 5} är sant. 
\begrepp{Indentering}{Högerjusterad eller indragen kod, vilket används i python för att markera vilken kod som tillhör ett kodblock. 
Exempelvis vilken kod som är inuti en if-sats.}

\subsection{Jämförelseoperatorer}
För att skapa villkor används jämförelseoperatorer, som till exempel:\\
\textbf{==} Är lika med (till exempel \texttt{x == 5})\\
\textbf{!=} Är inte lika med (till exempel \texttt{x != 5})\\
\textbf{>} Större än\\
\textbf{<} Mindre än\\
\textbf{>=} Större än eller lika med\\
\textbf{<=} Mindre än eller lika med\\

\observera{== betyder att vi jämför om två saker är likamed varandra. Jämför med ett likamedtecken (=) som används för att ge ett variabel ett värde (tilldelning).
Vi använder == när vi gör en jämförelse för att inte förväxla med variabeltilldelning.}

\begin{lstlisting}[title=Exempel med jämförelseoperatorer]
x = 8
if x != 5:
    print("x är inte lika med 5")
\end{lstlisting}

Här är villkoret sant eftersom \texttt{x = 8} inte är lika med \texttt{5}, så texten skrivs ut.

\subsection{\texttt{else} och \texttt{elif}}
Ibland vill vi hantera flera olika fall. Då kan vi använda \texttt{else} för att göra något när villkoret inte är sant, eller \texttt{elif} (som står för "else if") för att lägga till fler villkor.

\begin{lstlisting}[title=Exempel med else och elif]
x = 10
if x < 5:
    print("x är mindre än 5")
elif x == 10:
    print("x är exakt 10")
else:
    print("x är större än 5 men inte 10")
\end{lstlisting}

När detta körs, skrivs \texttt{x är exakt 10} ut, eftersom det andra villkoret (\texttt{x == 10}) är sant.
\observera{En elif körs bara om alla villkor ovanför är falska.}

\begrepp{else}{En \texttt{else}-sats körs när inget av de tidigare villkoren i en \texttt{if}-sats är sanna.}
\begrepp{elif}{En \texttt{elif}-sats används för att lägga till ytterligare villkor till en \texttt{if}-sats.}

\subsection{Övningar}
\begin{exercise}
Skriv en \texttt{if}-sats som kontrollerar om ett tal är större än 100. Om villkoret är sant, skriv ut \texttt{Talet är stort.}
\end{exercise}

\begin{solution}
\begin{lstlisting}
x = 150
if x > 100:
    print("Talet är stort.")
\end{lstlisting}
\end{solution}

\begin{exercise}
Använd en \texttt{if}-sats med \texttt{else}. Skriv ett program som kontrollerar om ett tal är negativt eller positivt, och skriver ut lämpligt meddelande.
\end{exercise}

\begin{solution}
\begin{lstlisting}
x = -10
if x >= 0:
    print("Talet är positivt.")
else:
    print("Talet är negativt.")
\end{lstlisting}
\end{solution}

\begin{exercise}
Skapa ett program som använder \texttt{elif} för att skriva ut olika meddelanden beroende på ålder:
- Om åldern är under 13, skriv \texttt{Du är ett barn.}
- Om åldern är mellan 13 och 19 (inklusive), skriv \texttt{Du är en tonåring.}
- Annars skriv \texttt{Du är vuxen.}
\end{exercise}

\begin{solution}
\begin{lstlisting}
age = 15
if age < 13:
    print("Du är ett barn.")
elif age <= 19:
    print("Du är en tonåring.")
else:
    print("Du är vuxen.")
\end{lstlisting}
\end{solution}

\secToExRef{examples:if}

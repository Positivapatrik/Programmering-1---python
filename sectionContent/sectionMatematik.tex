\section{Matematik i Python: \texttt{math}-modulen och specialoperatorer}
\label{section:math}
Python har inbyggda verktyg för matematiska beräkningar. Förutom grundläggande operatorer som \texttt{+}, \texttt{-}, \texttt{*}, och \texttt{/} finns fler avancerade funktioner i \texttt{math}-modulen. Här ska vi även titta på specialoperatorerna \texttt{**}, \texttt{\%}, och \texttt{//}.

\subsection{\texttt{math}-modulen}
\texttt{math}-modulen är en inbyggd modul i Python som erbjuder en mängd matematiska funktioner och konstanter.

\subsubsection{Importera \texttt{math}-modulen}
För att använda modulen måste den importeras i din kod:
\begin{lstlisting}[title=Exempel: Importera \texttt{math}]
import math
\end{lstlisting}

Efter att du har importerat \texttt{math} kan du använda dess funktioner.

\subsubsection{Vanliga funktioner i \texttt{math}}
Här är några av de mest använda funktionerna i \texttt{math}:
\begin{lstlisting}[title=Exempel på \texttt{math}-funktioner]
import math

# Roten ur ett tal
print(math.sqrt(16))  # 4.0

# Upphöjning till en viss potens
print(math.pow(2, 3))  # 8.0

# Heltalsdel av en division
print(math.floor(7.8))  # 7
print(math.ceil(7.2))   # 8

# Värdet av pi och e
print(math.pi)  # 3.141592653589793
print(math.e)   # 2.718281828459045
\end{lstlisting}

\begrepp{Modul}{En modul är ett bibliotek av funktioner och variabler som kan importeras och användas i ett program.}

\subsection{Specialoperatorer}
Python har flera operatorer som är användbara för matematiska beräkningar utöver de grundläggande.

\subsubsection{\texttt{**} (Exponentiering)}
Operatorn \texttt{**} används för att upphöja ett tal till en viss potens.
\begin{lstlisting}[title=Exempel på \texttt{**}]
print(2 ** 3)  # 8 (2 upphöjt till 3)
print(5 ** 2)  # 25 (5 upphöjt till 2)
\end{lstlisting}

\subsubsection{\texttt{\%} (Modulus)}
Operatorn \texttt{\%} returnerar resten vid en division.
\begin{lstlisting}[title=Exempel på \texttt{\%}]
print(10 % 3)  # 1 (10 dividerat med 3 ger resten 1)
print(15 % 4)  # 3 (15 dividerat med 4 ger resten 3)
\end{lstlisting}

\begrepp{Modulus}{En operation som returnerar resten vid division av två tal.}

\subsubsection{\texttt{//} (Heltalsdivision)}
Operatorn \texttt{//} utför division och returnerar endast heltalsdelen av resultatet.
\begin{lstlisting}[title=Exempel på \texttt{//}]
print(10 // 3)  # 3 (heltalsdelen av 10 dividerat med 3)
print(15 // 4)  # 3 (heltalsdelen av 15 dividerat med 4)
\end{lstlisting}

\begrepp{Heltalsdivision}{En operation som returnerar heltalsdelen av en division, utan decimaler.}

\subsection{Praktiskt exempel: Beräkna cirkelns area}
Här använder vi \texttt{math}-modulen och en av specialoperatorerna för att beräkna en cirkels area.
\begin{lstlisting}[title=Beräkna cirkelns area]
import math

# Fråga användaren efter radien
radie = float(input("Ange radien: "))

# Beräkna arean
area = math.pi * (radie ** 2)
print("Cirkelns area är:", area)
\end{lstlisting}

\subsection{Övningar}
\begin{exercise}
Skriv ett program som använder \texttt{math.sqrt()} för att beräkna kvadratroten av ett tal som användaren anger.
\end{exercise}

\begin{solution}
\begin{lstlisting}
import math

# Fråga användaren efter ett tal
tal = float(input("Ange ett tal: "))

# Beräkna och visa kvadratroten
print("Kvadratroten av", tal, "är:", math.sqrt(tal))
\end{lstlisting}
\end{solution}

\begin{exercise}
Skriv ett program som använder \texttt{\%} för att avgöra om ett tal som användaren anger är jämnt eller udda.
\end{exercise}

\begin{solution}
\begin{lstlisting}
# Fråga användaren efter ett tal
tal = int(input("Ange ett tal: "))

# Kontrollera om talet är jämnt eller udda
if tal % 2 == 0:
    print("Talet är jämnt.")
else:
    print("Talet är udda.")
\end{lstlisting}
\end{solution}

\begin{exercise}
Skriv ett program som frågar användaren efter två tal och skriver ut resultatet av \texttt{//} och \texttt{\%}.
\end{exercise}

\begin{solution}
\begin{lstlisting}
# Fråga användaren efter två tal
tal1 = int(input("Ange det första talet: "))
tal2 = int(input("Ange det andra talet: "))

# Beräkna heltalsdivision och resten
print("Heltalsdivision:", tal1 // tal2)
print("Rest:", tal1 % tal2)
\end{lstlisting}
\end{solution}

\secToExRef{examples:math}

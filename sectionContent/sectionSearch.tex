\section{Sökning}
\label{section:search}
När vi arbetar med data är det vanligt att vi behöver hitta ett specifikt värde i en samling, som en lista. Det finns olika sätt att göra detta, beroende på hur data är organiserad. Två grundläggande sökalgoritmer är \textbf{linjär sökning} och \textbf{binär sökning}.

\subsection{Linjär sökning}
Linjär sökning är den enklaste sökalgoritmen. Den går igenom varje element i listan, ett i taget, tills det hittar det sökta värdet eller når slutet av listan.

\begin{itemize}
    \item Fördel: Fungerar på både sorterade och osorterade listor.
    \item Nackdel: Kan vara långsam för långa listor eftersom varje element måste kontrolleras.
\end{itemize}

\subsubsection*{Kodexempel}
Här är ett exempel på linjär sökning i Python:
\begin{lstlisting}[title=Linjär sökning]
def linjar_sokning(lista, mål):
    for index, värde i enumerate(lista):
        if värde == mål:
            return index  # Returnerar positionen
    return -1  # Returnerar -1 om värdet inte finns

# Exempel
data = [10, 20, 30, 40, 50]
print(linjar_sokning(data, 30))  # Output: 2
\end{lstlisting}

\subsection{Binär sökning}
Binär sökning är en mer effektiv algoritm för att hitta ett värde i en sorterad lista. Algoritmen delar listan på mitten och avgör om det sökta värdet är mindre eller större än mittpunkten. Därefter upprepas processen på den relevanta halvan av listan.

\begin{itemize}
    \item Fördel: Mycket snabbare än linjär sökning för stora listor.
    \item Nackdel: Kräver att listan är sorterad.
\end{itemize}

\subsubsection*{Hur fungerar binär sökning?}
\begin{enumerate}
    \item Dela listan på mitten.
    \item Kontrollera mittpunkten:
        \begin{itemize}
            \item Om värdet är det sökta, är vi klara.
            \item Om värdet är mindre än det sökta, leta i den högra halvan.
            \item Om värdet är större, leta i den vänstra halvan.
        \end{itemize}
    \item Upprepa tills värdet hittas eller listan är tom.
\end{enumerate}

\subsubsection*{Kodexempel}
Här är ett exempel på binär sökning i Python:
\begin{lstlisting}[title=Binär sökning]
def binar_sokning(lista, mål):
    vänster, höger = 0, len(lista) - 1
    while vänster <= höger:
        mitten = (vänster + höger) // 2
        if lista[mitten] == mål:
            return mitten
        elif lista[mitten] < mål:
            vänster = mitten + 1
        else:
            höger = mitten - 1
    return -1  # Returnerar -1 om värdet inte finns

# Exempel
data = [10, 20, 30, 40, 50]
print(binar_sokning(data, 30))  # Output: 2
\end{lstlisting}

\subsection{Jämförelse mellan linjär och binär sökning}
\begin{table}[h!]
\centering
\begin{tabular}{|l|l|l|}
\hline
\textbf{Egenskap} & \textbf{Linjär sökning} & \textbf{Binär sökning} \\ \hline
Krav på sortering  & Nej                      & Ja                     \\ \hline
Effektivitet       & Långsam för långa listor & Snabb för långa listor \\ \hline
Komplexitet        & \(O(n)\)                & \(O(\log n)\)          \\ \hline
\end{tabular}
\caption{Jämförelse mellan linjär och binär sökning.}
\end{table}

\subsection{Övning}
\begin{exercise}
Skriv en Python-funktion som implementerar linjär sökning och använd den för att hitta värdet 25 i listan \texttt{[5, 15, 25, 35, 45]}.
\end{exercise}

\begin{exercise}
Använd kodexemplet för binär sökning för att hitta värdet 50 i listan \texttt{[10, 20, 30, 40, 50]}.
\end{exercise}

\secToExRef{examples:search}

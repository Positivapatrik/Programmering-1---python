
\newtcolorbox[auto counter, number within=section]{rconceptbox}[2][]{
    size=small,
    sharp corners,
    right=1cm,
    colframe=orange,
    colback=orange!10,
    coltitle=black,
%    floatplacement=ht,float,
    title=#2,
    before=\vspace{1em}, % Mellanrum ovanför rutan
    after=\vspace{1em},  % Mellanrum under rutan
    #1,
}

\newtcolorbox{outputbox}[2][]{
    size=small,
    sharp corners,
    right=1cm,
%    colframe=gray!10,
    colback=blue!5,
    coltitle=blue!10,
%    floatplacement=ht,float,
    title=#2,
    before=\vspace{1em}, % Mellanrum ovanför rutan
    after=\vspace{1em},  % Mellanrum under rutan
    #1,
}


\newtcolorbox{observerabox}[1][]{
    size=small,
    sharp corners,
    right=1cm,
    colframe=OliveGreen,
    colback=OliveGreen!10,
    coltitle=black,
    title=Observera!,
    before=\vspace{1em}, % Mellanrum ovanför rutan
    after=\vspace{1em},  % Mellanrum under rutan
    #1,
}

\newtcolorbox{refbox}[1][]{
    size=small,
    sharp corners,
    right=1cm,
    colframe=Blue,
    colback=Blue!10,
    coltitle=black,
    before=\vspace{1em}, % Mellanrum ovanför rutan
    after=\vspace{1em},  % Mellanrum under rutan
    #1,
}

\lstset{
    literate={å}{{\r{a}}}1
             {ä}{{\"a}}1
             {ö}{{\"o}}1
             {Å}{{\r{A}}}1
             {Ä}{{\"A}}1
             {Ö}{{\"O}}1
}
% Anpassa utseendet på kod
\lstset{
    basicstyle=\ttfamily\small,
    keywordstyle=\color{blue},
    commentstyle=\color{green!50!black},
    stringstyle=\color{red},
    backgroundcolor=\color{gray!10},
    frame=single,
    numbers=left,
    numberstyle=\tiny\color{gray},
    breaklines=true,
    resetmargins=true,
    captionpos=b,
    language=Python,
    showstringspaces=false,
    tabsize=2,
    upquote=true,
    floatplacement=ht,float
}

\newcommand{\begrepp}[2]{
	\begin{rconceptbox}{#1}#2\end{rconceptbox}
}
\newcommand{\observera}[1]{
	\begin{observerabox}{}#1\end{observerabox}
}
\newcommand{\pythonoutput}[2]{
	\begin{outputbox}{#1}#2 \end{outputbox}
}
\newcommand{\exToSecRef}[1]{
Exempel från avsnitt \ref{#1}.
}
\newcommand{\secToExRef}[1]{
\begin{refbox}För fler kodexempel se appendix \ref{#1}\end{refbox}
}


\title{Programmering 1 med Python}
\author{Patrik Berggren}

\hypersetup{colorlinks}

\DeclareExerciseHeadingTemplate{custom}
  {\section{\XSIMexpandcode{\XSIMtranslate{default-heading}}}}

\DeclareExerciseEnvironmentTemplate{custom}
  {%
    \IfInsideSolutionTF
      {\label{sol:\ExerciseID}}
      {\label{ex:\ExerciseID}}%
    \paragraph*
      {%
        \XSIMmixedcase{\GetExerciseName}%
        \IfInsideSolutionTF
          {
            till \GetExerciseParameter{exercise-name}%
            ~\GetExerciseProperty{counter}%
            ~(Från sida~\pageref{ex:\ExerciseID})
          }
          {%
            ~\GetExerciseProperty{counter}%
%            \GetExercisePropertyT{subtitle}{~(\PropertyValue)}
          }
      }
      \noindent
  }
  {%
    \IfInsideSolutionF
      {\par\leavevmode\hfill
       Lösningar på sida~\pageref{sol:\ExerciseID}~$\blacktriangleleft$}%
  }
\xsimsetup{
  exercise/name = Uppgift,
  solution/name = Lösning,
  exercise/template = custom ,
  solution/template = custom ,
  print-solutions/headings-template = custom
}
\DeclareExerciseHeadingTemplate{solution}{%
    \section{ Lösningsförslag}
}




%For flowcharts
\usepackage{tikz}
\usetikzlibrary{shapes.geometric, arrows}

\tikzstyle{startstop} = [rectangle, rounded corners, 
minimum width=3cm, 
minimum height=1cm,
text centered, 
draw=black, 
fill=red!30]

\tikzstyle{io} = [trapezium, 
trapezium stretches=true, % A later addition
trapezium left angle=70, 
trapezium right angle=110, 
minimum width=3cm, 
minimum height=1cm, text centered, 
draw=black, fill=blue!30]

\tikzstyle{process} = [rectangle, 
minimum width=3cm, 
minimum height=1cm, 
text centered, 
text width=3cm, 
draw=black, 
fill=orange!30]

\tikzstyle{decision} = [diamond, 
minimum width=3cm, 
minimum height=1cm,
%yshift=-0.5cm,
aspect=3,
text centered, 
draw=black, 
fill=green!30]
\tikzstyle{arrow} = [thick,->,>=stealth]
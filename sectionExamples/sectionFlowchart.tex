\section{Kodexempel: Flödesdiagram}
\label{examples:flowchart}
\exToSecRef{section:flowchart}
\subsection*{Exempel 1: Enkel sekvens}

\begin{tabular*}{\linewidth}{@{\extracolsep{\fill}} p{0.45\linewidth} | p{0.5\linewidth}}
\textbf{Flödesdiagram:} & \textbf{Python-kod:} \\
\hline

% Flödesdiagram
\raisebox{-\totalheight}{%
\begin{tikzpicture}[node distance=2cm]
\node (start) [startstop] {Start};
\node (input) [io, below of=start] {Läs tal};
\node (process) [process, below of=input] {Addera 5};
\node (output) [io, below of=process] {Skriv resultat};
\node (stop) [startstop, below of=output] {Stop};

\draw [arrow] (start) -- (input);
\draw [arrow] (input) -- (process);
\draw [arrow] (process) -- (output);
\draw [arrow] (output) -- (stop);
\end{tikzpicture}
}
&

% Python-kod
\raggedright
\begin{lstlisting}[xleftmargin=1.5em]
# Läs ett tal från användaren
nummer = int(input("Ange ett tal: "))

# Lägg till 5
resultat = nummer + 5

# Skriv ut resultatet
print("Resultatet är:", resultat)
\end{lstlisting}
\vspace{0.5em}
Enkelt program som läser in ett tal från användaren, adderar 5 till det, och skriver ut resultatet igen.
\\
\end{tabular*}


\subsection*{Exempel 2: Beslutsstruktur}
\begin{tabular*}{\linewidth}{@{\extracolsep{\fill}} p{0.45\linewidth} | p{0.5\linewidth}}
\textbf{Flödesdiagram:} & \textbf{Python-kod:} \\
\hline

% Flödesdiagram
\raisebox{-\totalheight}{%
\begin{tikzpicture}[node distance=2cm]
\node (start) [startstop] {Start};
\node (input) [io, below of=start] {Läs tal};
\node (decision) [decision, below of=input] {tal > 0?};
\node (process) [process, right of=decision,xshift=2cm] {Addera 5};
\node (output1) [io, below of=process] {Skriv resultat};
\node (output2) [io, below of=decision, yshift=-1cm] {Skriv negativt};
\node (stop) [startstop, below of=output2] {Stop};

\draw [arrow] (start) -- (input);
\draw [arrow] (input) -- (decision);
\draw [arrow] (decision.east) -- node[anchor=south] {ja} (process.west);
\draw [arrow] (process) -- (output1);
\draw [arrow] (decision.south) -- node[anchor=east] {nej} (output2.north);
\draw [arrow] (output1.south) |- (stop.east);
\draw [arrow] (output2) -- (stop);
\end{tikzpicture}
}
&

% Python-kod
\raggedright
\begin{lstlisting}[xleftmargin=1.5em]
# Läs ett tal från användaren
nummer = int(input("Ange ett tal: "))

if nummer > 0:
    # Addera 5 om talet är positivt
    resultat = nummer + 5
    print("Resultatet är:", resultat)
else:
    # Annars skriv "negativt"
    print("Talet är negativt")
\end{lstlisting}
\vspace{0.5em}
Användaren skriver in ett tal. Om det är större än 0 adderas fem till talet, och resultatet skrivs ut.
Annars skrivs endast ut att talet är negativt utan att addera något.
\\
\end{tabular*}


\subsection*{Exempel 3: While-loop med villkor}
\begin{tabular*}{\linewidth}{@{\extracolsep{\fill}} p{0.55\linewidth} | p{0.45\linewidth}}
\textbf{Flödesdiagram:} & \textbf{Python-kod:} \\
\hline

% Flödesdiagram
\raisebox{-\totalheight}{%
\begin{tikzpicture}[node distance=2cm]
\node (start) [startstop] {Start};
\node (summainit) [process, below of=start] {Sätt summa till 0};
\node (talinit) [process, below of=summainit] {Sätt tal till 0};
\node (decision) [decision, below of=talinit] {summa < 1000?};
\node (summera) [process, below of=decision] {summa <- summa + tal};
\node (ökatal) [process, below of=summera] {Öka tal med 1};
\node (stop) [startstop, right of=decision, xshift=2.5cm] {Stop};

\draw [arrow] (start) -- (summainit);
\draw [arrow] (summainit) -- (talinit);
\draw [arrow] (talinit) -- (decision);
\draw [arrow] (decision.east) -- node[anchor=south] {ja} (stop.west);
\draw [arrow] (decision.south) -- node[anchor=east] {nej} (summera);
\draw [arrow] (summera) -- (ökatal);
\draw [arrow] (ökatal.west) -- ++(-1,0) -- ++(0,4) -- (decision.west);

%\draw [arrow] (ökatal) -- (decision.west);

\end{tikzpicture}
}
&

% Python-kod
\raggedright
\begin{lstlisting}[caption={Summera positiva tal}, xleftmargin=1.5em]
# Räknar 1+2+3+4+5... så länge summan är mindre än 1000
summa = 0
tal = 1
while summa < 1000:
    summa = summa + tal

print("Summa 1+2+3+...+",tal,">1000")

\end{lstlisting}
\vspace{0.5em}
Räknar ut vilket tal vi behöver räkna till så att $1+2+3+\cdots+tal>1000$.
\\
\end{tabular*}
\newpage
\subsection*{Exempel 4: Kombinerad if-sats och while-loop}
\begin{tabular*}{\linewidth}{@{\extracolsep{\fill}} p{0.55\linewidth} | p{0.48\linewidth}}
\textbf{Flödesdiagram:} & \textbf{Python-kod:} \\
\hline

% Flödesdiagram
\raisebox{-\totalheight}{%
\begin{tikzpicture}[node distance=2cm]
\node (start) [startstop] {Start};
\node (antalinit) [process, below of=start] {Antal försök <- 0};

\node (input) [io, below of=antalinit] {Läs lösenord};
\node (decision1) [decision, below of=input] {antal försök < 10?};
\node (antalöka) [process, below of=decision1] {Öka antal försök med 1};

\node (decision2) [decision, below of=antalöka] {lösenord == korrekt?};
\node (denied) [io, right of=antalöka, xshift=2cm] {Skriv ut ''Nekad''};
\node (access) [io, below of=decision2] {Skriv ut ''Beviljad''};
\node (maxtries) [io, below of=access] {Skriv ut ''Inga fler försök tillåts''};


\node (stop) [startstop, below of=maxtries] {Stop};

\draw [arrow] (start) -- (antalinit);
\draw [arrow] (antalinit) -- (input);
\draw [arrow] (input) -- (decision1);
\draw [arrow] (decision1.south) -- node[anchor=east] {ja} (antalöka.north);

\draw [arrow] (antalöka) -- (decision2);
\draw [arrow] (decision2.south) -- node[anchor=east] {ja} (access.north);

\draw [arrow] (decision1.west) -- node[anchor=south] {Nej} ++(-1,0) -- ++(0,-8) -- (maxtries.west);
\draw [arrow] (maxtries) -- (stop);

\draw [arrow] (access.east) -- ++(1.5,0) -- ++(0,-4) -- (stop.east);


\draw [arrow] (denied.north) -- ++(0,3.5) -- (input.east);
\draw [arrow] (decision2.east) -- node[anchor=south] {nej} ++(1.25,0) -- (denied.south);


\end{tikzpicture}
}
&

% Python-kod
\raggedright
\begin{lstlisting}[xleftmargin=1.5em]
# Enkel lösenordskontroll
korrekt_losenord = "python123"
antal_försök=0

while antal_försök<10:
    losenord = input("Ange lösenord: ")
    if losenord == korrekt_losenord:
        print("Beviljad!")
        break
    else:
        print("Nekad")
    antal_försök = antal_försök + 1
if antal_försök == 10:
  print("Inga fler försök tillåts")
\end{lstlisting}
\vspace{0.5em}
Låter användaren mata in ett lösenord tills dess att användaren skriver in
''python123''. Programmet skriver ut meddelanden vid varje försök som säger
om lösenordet var rätt eller inte. Programmet ger max 10 försök till användaren.
Notera att i flödesdiagrammet finns ingen direkt motsvarighet till if-satsen efter loopen.
Det beror på att flödesdiagramet beskriver vad programmet gör logiskt och inte i en direktöversättning. 
Eftersom loopen i flödesdiagramet avslutas genom \textbf{Antal försök > 10?} behövs inte en extra koll.
\\
\end{tabular*}


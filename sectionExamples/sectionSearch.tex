\section{Kodexempel: Linjär och Binär Sökning}
\label{examples:search}
\exToSecRef{section:search}
\subsection*{Exempel 1: Linjär sökning i en lista}

\begin{lstlisting}[title=Linjär sökning i en lista]
def linjar_sokning(lista, mål):
    for index, värde i enumerate(lista):
        if värde == mål:
            return index  # Returnera index om målet hittas
    return -1  # Returnera -1 om målet inte finns

tal_lista = [10, 20, 30, 40, 50]
mål = 30
resultat = linjar_sokning(tal_lista, mål)
print(f"Elementet finns vid index: {resultat}")
# Output: Elementet finns vid index: 2
\end{lstlisting}

\subsection*{Exempel 2: Linjär sökning i en dictionary}

\begin{lstlisting}[title=Linjär sökning i en dictionary]
elever = {"Anna": 85, "Björn": 75, "Cecilia": 90, "David": 80}

def linjar_sokning_dict(dictionary, mål):
    for nyckel, värde i dictionary.items():
        if värde == mål:
            return nyckel
    return None

mål = 90
resultat = linjar_sokning_dict(elever, mål)
print(f"Eleven med poäng {mål} är: {resultat}")
# Output: Eleven med poäng 90 är: Cecilia
\end{lstlisting}

\subsection*{Exempel 3: Binär sökning i en sorterad lista}

\begin{lstlisting}[title=Binär sökning i en sorterad lista]
def binar_sokning(lista, mål):
    vänster, höger = 0, len(lista) - 1
    while vänster <= höger:
        mitten = (vänster + höger) // 2
        if lista[mitten] == mål:
            return mitten
        elif lista[mitten] < mål:
            vänster = mitten + 1
        else:
            höger = mitten - 1
    return -1

tal_lista = [10, 20, 30, 40, 50]
mål = 40
resultat = binar_sokning(tal_lista, mål)
print(f"Elementet finns vid index: {resultat}")
# Output: Elementet finns vid index: 3
\end{lstlisting}

\subsection*{Exempel 4: Visualisering av binär sökning}

\begin{lstlisting}[title=Binär sökning - Visualisering]
def binar_sokning_visualisering(lista, mål):
    vänster, höger = 0, len(lista) - 1
    steg = 0
    while vänster <= höger:
        mitten = (vänster + höger) // 2
        print(f"Steg {steg}: Vänster={vänster}, Höger={höger}, Mitten={mitten}")
        if lista[mitten] == mål:
            return mitten
        elif lista[mitten] < mål:
            vänster = mitten + 1
        else:
            höger = mitten - 1
        steg += 1
    return -1

tal_lista = [10, 20, 30, 40, 50]
mål = 40
binar_sokning_visualisering(tal_lista, mål)
# Output:
# Steg 0: Vänster=0, Höger=4, Mitten=2
# Steg 1: Vänster=3, Höger=4, Mitten=3
# Output: Elementet finns vid index: 3
\end{lstlisting}

\subsection*{Exempel 5: Tidskomplexitet}

\begin{rconceptbox}{Tidskomplexitet för sökning}{
\textbf{Linjär sökning:} $O(n)$, där $n$ är antalet element i listan.\\
\textbf{Binär sökning:} $O(\log n)$, men listan måste vara sorterad.
}
\end{rconceptbox}

\subsection*{Exempel 6: Binär sökning med rekursion}

\begin{lstlisting}[title=Binär sökning - Rekursiv implementation]
def binar_sokning_rekursiv(lista, mål, vänster, höger):
    if vänster > höger:
        return -1
    mitten = (vänster + höger) // 2
    if lista[mitten] == mål:
        return mitten
    elif lista[mitten] < mål:
        return binar_sokning_rekursiv(lista, mål, mitten + 1, höger)
    else:
        return binar_sokning_rekursiv(lista, mål, vänster, mitten - 1)

tal_lista = [10, 20, 30, 40, 50]
mål = 50
resultat = binar_sokning_rekursiv(tal_lista, mål, 0, len(tal_lista) - 1)
print(f"Elementet finns vid index: {resultat}")
# Output: Elementet finns vid index: 4
\end{lstlisting}

\subsection*{Praktiska tips}
- Linjär sökning fungerar för osorterade listor eller dictionaries.
- Binär sökning kräver en sorterad lista och är betydligt snabbare för större datamängder.

\section{Kodexempel: Kodkommentarer}
\label{examples:comments}
\exToSecRef{section:comments}
\subsection*{Grunderna i kommentarer}


\begin{lstlisting}[title=Exempel 1: Enkel kommentar med \texttt{\#}]
# Detta är en kommentar. Den körs inte av programmet.
print("Hej!")  # Detta skriver ut "Hej!" till skärmen.
# Kommentarer kan användas för att förklara vad koden gör.
\end{lstlisting}

\begin{lstlisting}[title=Exempel 2: Multiradkommentarer med trippelcitat (\texttt{"""})]
"""
Detta är en kommentar som kan sträcka sig
över flera rader. Den används ofta för att
beskriva ett program eller en funktion.
"""
print("Hej från ett program med kommentarer!")
\end{lstlisting}

\begin{lstlisting}[title=Exempel 3: Kommentarer för felsökning]
# Följande rad är tillfälligt avstängd för att felsöka programmet.
# print("Den här raden är inaktiverad.")
print("Den här raden körs.")
# Kommenterad kod kan aktiveras igen om det behövs.
\end{lstlisting}

\subsection*{God kommentarpraxis}

\begin{lstlisting}[title=Exempel 4: Klara och tydliga kommentarer]
# Beräknar summan av två tal och skriver ut resultatet
a = 5
b = 7
summan = a + b
print("Summan är:", summan)
\end{lstlisting}

\begin{lstlisting}[title=Exempel 5: Kommentera inte självklarheter]
# Dåligt exempel:
a = 5  # Sätt a till 5
b = 7  # Sätt b till 7
# Det här är onödiga kommentarer eftersom koden redan förklarar sig själv.
\end{lstlisting}

\begin{lstlisting}[title=Exempel 6: Förklara varför och inte bara vad]
# Dåligt exempel:
a = 5  # Sätt a till 5
# Bra exempel:
a = 5  # Antalet användare som får tillgång samtidigt
\end{lstlisting}

\subsection*{\texttt{TODO}-kommentarer}

\begin{lstlisting}[title=Exempel 7: Använd TODO-kommentarer för framtida uppgifter]
# TODO: Lägg till inmatning för användaren
a = 10
b = 20
print("Summan är:", a + b)
\end{lstlisting}

\subsection*{Tillämpning av kommentarer}

\begin{lstlisting}[title=Exempel 8: Kombinera olika typer av kommentarer]
"""
Detta program läser in två tal från användaren,
beräknar summan och visar resultatet.
"""
# Läser in två tal från användaren
a = int(input("Ange det första talet: "))  # Konverterar till heltal
b = int(input("Ange det andra talet: "))  # Konverterar till heltal

# Beräkna och skriv ut summan
summan = a + b
print("Summan är:", summan)  # Skriver ut resultatet
\end{lstlisting}

\subsection*{Fler exempel}

\begin{lstlisting}[title=Exempel 9: Kommentarer för komplexa delar av koden]
# Kontrollera om ett tal är jämnt eller udda
number = 7
# Om resten av division med 2 är 0, är talet jämnt
if number % 2 == 0:
    print("Talet är jämnt.")
else:
    print("Talet är udda.")
\end{lstlisting}

\begin{lstlisting}[title=Exempel 10: Sätt din kod i kontext med kommentarer]
"""
Detta är en enkel kalkylator som kan addera, subtrahera,
multiplicera och dividera två tal. Just nu är funktionerna
för multiplikation och division inte implementerade.
"""
# TODO: Implementera multiplikation och division
a = 10
b = 5

print("Summan är:", a + b)  # Addition
print("Skillnaden är:", a - b)  # Subtraktion
\end{lstlisting}
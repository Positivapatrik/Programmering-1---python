\section{Kodexempel: Listor}
\label{examples:lists}
\exToSecRef{section:lists}
\subsection*{Skapa och använda listor}

\begin{lstlisting}[title=Exempel 1: Skapa en tom lista]
# En tom lista kan skapas med []
min_lista = []
print("Tom lista:", min_lista)  # []
\end{lstlisting}

\begin{lstlisting}[title=Exempel 2: Skapa en lista med tal]
# En lista med heltal
tal_lista = [1, 2, 3, 4, 5]
print("Lista med tal:", tal_lista)  # [1, 2, 3, 4, 5]
\end{lstlisting}

\begin{lstlisting}[title=Exempel 3: Skapa en lista med strängar]
# En lista med textsträngar
sträng_lista = ["äpple", "banan", "körsbär"]
print("Lista med strängar:", sträng_lista)  # ["äpple", "banan", "körsbär"]
\end{lstlisting}

\begin{lstlisting}[title=Exempel 4: Blandad lista]
# En lista kan innehålla olika datatyper
blandad_lista = [10, "hej", 3.14, True]
print("Blandad lista:", blandad_lista)  # [10, "hej", 3.14, True]
\end{lstlisting}

\subsection*{Åtkomst med index}

\begin{lstlisting}[title=Exempel 5: Hämta ett element med index]
# Hämta det första elementet (index 0)
frukt_lista = ["äpple", "banan", "körsbär"]
print("Första frukten:", frukt_lista[0])  # "äpple"
\end{lstlisting}

\begin{lstlisting}[title=Exempel 6: Hämta det sista elementet]
# Hämta det sista elementet med negativt index
print("Sista frukten:", frukt_lista[-1])  # "körsbär"
\end{lstlisting}

\begin{lstlisting}[title=Exempel 7: Indexeringsfel (out-of-bounds)]
# Försök att komma åt ett index utanför listans storlek
try:
    print(frukt_lista[3])  # IndexError
except IndexError:
    print("Fel: Index utanför listans storlek.")
\end{lstlisting}

\subsection*{Modifiera listor}

\begin{lstlisting}[title=Exempel 8: Lägg till ett element med append()]
# Lägg till ett nytt element i slutet av listan
frukt_lista.append("druva")
print("Efter append:", frukt_lista)  # ["äpple", "banan", "körsbär", "druva"]
\end{lstlisting}

\begin{lstlisting}[title=Exempel 9: Ta bort ett element med remove()]
# Ta bort ett element med remove()
frukt_lista.remove("banan")
print("Efter remove:", frukt_lista)  # ["äpple", "körsbär", "druva"]
\end{lstlisting}

\begin{lstlisting}[title=Exempel 10: Sortera en lista med sort()]
# Sortera en lista av tal
nummer_lista = [5, 2, 9, 1, 7]
nummer_lista.sort()
print("Sorterad lista:", nummer_lista)  # [1, 2, 5, 7, 9]
\end{lstlisting}

\subsection*{Listaegenskaper}

\begin{lstlisting}[title=Exempel 11: Antal element med len()]
# Räkna antalet element i en lista
antal = len(frukt_lista)
print("Antal frukter:", antal)  # 3
\end{lstlisting}

\begin{lstlisting}[title=Exempel 12: Tom lista med len()]
# En tom lista har längden 0
tom_lista = []
print("Längden av en tom lista:", len(tom_lista))  # 0
\end{lstlisting}

\section{Kodexempel: Variabler och sekventiell exekvering}
\label{examples:variables}
\exToSecRef{section:variables}
\subsection*{Grunderna för variabler}

\begin{lstlisting}[title=Exempel 1: Skapa en variabel]
name = "Alice"
print(name)
# Output: Alice
\end{lstlisting}

\begin{lstlisting}[title=Exempel 2: Variabel med heltal]
age = 25
print(age)
# Output: 25
\end{lstlisting}

\begin{lstlisting}[title=Exempel 3: Variabel med flyttal]
price = 19.99
print(price)
# Output: 19.99
\end{lstlisting}

\subsection*{Datatyper och deras användning}

\begin{lstlisting}[title=Exempel 4: Kombinera variabler och text]
name = "Bob"
age = 30
print("Namn:", name)
print("Ålder:", age)
# Output:
# Namn: Bob
# Ålder: 30
\end{lstlisting}

\begin{lstlisting}[title=Exempel 5: Konkatenering av strängar]
first_name = "Alice"
last_name = "Smith"
full_name = first_name + " " + last_name
print(full_name)
# Output: Alice Smith
\end{lstlisting}

\begin{lstlisting}[title=Exempel 6: Implicit datakonvertering är inte tillåten]
age = 20
print("Jag är " + age + " år gammal.")
# Ger ett TypeError! (sträng + int fungerar inte)
\end{lstlisting}

\begin{lstlisting}[title=Exempel 7: Konvertera till sträng för att undvika fel]
age = 20
print("Jag är " + str(age) + " år gammal.")
# Output: Jag är 20 år gammal.
\end{lstlisting}

\subsection*{Sekventiell exekvering}

\begin{lstlisting}[title=Exempel 8: Variabler kan ändra värde]
counter = 0
print(counter)  # Output: 0
counter = counter + 1
print(counter)  # Output: 1
\end{lstlisting}

\begin{lstlisting}[title=Exempel 9: Beräkningar med variabler]
a = 10
b = 3
result = a * b
print("Resultat:", result)
# Output: Resultat: 30
\end{lstlisting}

\begin{lstlisting}[title=Exempel 10: Variabler beroende av varandra]
x = 5
y = x + 2
print(y)  # Output: 7
x = 10
print(y)  # Output: 7 (ändringen av x påverkar inte y)
\end{lstlisting}

\subsection*{Blandade exempel}

\begin{lstlisting}[title=Exempel 11: Multiplicera text]
word = "Hej"
print(word * 3)
# Output: HejHejHej
\end{lstlisting}

\begin{lstlisting}[title=Exempel 12: Variabeln används innan den skapas]
print(name)
# Ger ett NameError! (variabeln är inte definierad)
\end{lstlisting}

\begin{lstlisting}[title=Exempel 13: Variabelns värde ändras stegvis]
balance = 100
balance = balance - 20
balance = balance + 50
print(balance)
# Output: 130
\end{lstlisting}

\begin{lstlisting}[title=Exempel 14: Variabler med flyttal och heltal tillsammans]
x = 10
y = 3.5
print("Summan är:", x + y)
# Output: Summan är: 13.5
\end{lstlisting}

\begin{lstlisting}[title=Exempel 15: Använda input och variabel tillsammans]
name = input("Vad heter du? ")
print("Hej, " + name + "!")
\end{lstlisting}

\begin{lstlisting}[title=Exempel 16: Avrundning med variabel]
pi = 3.14159
rounded_pi = round(pi, 2)
print("Pi avrundat:", rounded_pi)
# Output: Pi avrundat: 3.14
\end{lstlisting}

\begin{lstlisting}[title=Exempel 17: Kontrollera en variabels datatyp]
value = 42
print(type(value))
# Output: <class 'int'>
\end{lstlisting}

\begin{lstlisting}[title=Exempel 18: Multiplicera och uppdatera samtidigt]
x = 4
x *= 3  # Samma som x = x * 3
print(x)
# Output: 12
\end{lstlisting}

\begin{lstlisting}[title=Exempel 19: Förväxla inte namn på variabler]
total = 100
totaal = 200  # Felstavning skapar en ny variabel!
print(total)   # Output: 100
print(totaal)  # Output: 200
\end{lstlisting}

\begin{lstlisting}[title=Exempel 20: Variabler i längre beräkningar]
width = 5
height = 10
area = width * height
perimeter = 2 * (width + height)
print("Area:", area)
print("Omkrets:", perimeter)
# Output:
# Area: 50
# Omkrets: 30
\end{lstlisting}

\section{Kodexempel: If-satser (Beslutsfattande)}
\label{examples:if}
\exToSecRef{section:if}
\subsection*{Grunderna för if-satser}

\begin{lstlisting}[title=Exempel 1: En enkel if-sats]
x = 10
if x > 5:
    print("x är större än 5")
# Output: x är större än 5
\end{lstlisting}

\begin{lstlisting}[title=Exempel 2: Kontrollera likhet]
y = 8
if y == 8:
    print("y är lika med 8")
# Output: y är lika med 8
\end{lstlisting}

\begin{lstlisting}[title=Exempel 3: Använda else]
age = 17
if age >= 18:
    print("Du är vuxen.")
else:
    print("Du är inte vuxen.")
# Output: Du är inte vuxen.
\end{lstlisting}

\subsection*{elif och flera villkor}

\begin{lstlisting}[title=Exempel 4: Använda elif för fler alternativ]
temperature = 15
if temperature > 25:
    print("Det är varmt.")
elif temperature > 10:
    print("Det är svalt.")
else:
    print("Det är kallt.")
# Output: Det är svalt.
\end{lstlisting}

\begin{lstlisting}[title=Exempel 5: If-elif-else med gränsvärden]
score = 85
if score >= 90:
    print("Betyg: A")
elif score >= 75:
    print("Betyg: B")
elif score >= 60:
    print("Betyg: C")
else:
    print("Betyg: F")
# Output: Betyg: B
\end{lstlisting}

\subsection*{Indentering är viktigt}

\begin{lstlisting}[title=Exempel 6: Felaktig indentering]
x = 10
if x > 5:
print("x är större än 5")  # Ger ett IndentationError
\end{lstlisting}

\begin{lstlisting}[title=Exempel 7: Korrekt indentering]
x = 10
if x > 5:
    print("x är större än 5")  # Rätt indentering
# Output: x är större än 5
\end{lstlisting}

\subsection*{Jämförelseoperationer}

\begin{lstlisting}[title=Exempel 8: Kontrollera större än eller lika med]
x = 12
if x >= 10:
    print("x är minst 10")
# Output: x är minst 10
\end{lstlisting}

\begin{lstlisting}[title=Exempel 9: Kontrollera mindre än]
y = 7
if y < 10:
    print("y är mindre än 10")
# Output: y är mindre än 10
\end{lstlisting}

\begin{lstlisting}[title=Exempel 10: Kontrollera olika värden]
z = 15
if z != 20:
    print("z är inte 20")
# Output: z är inte 20
\end{lstlisting}

\subsection*{Blandade exempel}

\begin{lstlisting}[title=Exempel 11: Använda if för att hitta det största värdet]
a = 12
b = 9
if a > b:
    print("a är större än b")
else:
    print("b är större än eller lika med a")
# Output: a är större än b
\end{lstlisting}

\begin{lstlisting}[title=Exempel 12: If-sats med variabler i flera steg]
balance = 100
withdrawal = 120
if withdrawal <= balance:
    print("Uttaget är godkänt")
else:
    print("Otillräckligt saldo")
# Output: Otillräckligt saldo
\end{lstlisting}

\begin{lstlisting}[title=Exempel 13: Kontrollera om ett tal är jämnt]
number = 6
if number % 2 == 0:
    print("Talet är jämnt")
else:
    print("Talet är udda")
# Output: Talet är jämnt
\end{lstlisting}

\begin{lstlisting}[title=Exempel 14: If-sats för flera steg av diskontering]
price = 200
if price > 100:
    discount = 20
else:
    discount = 10
print("Rabatten är:", discount, "kr")
# Output: Rabatten är: 20 kr
\end{lstlisting}

\begin{lstlisting}[title=Exempel 15: Kontrollera intervall med if-satser]
x = 15
if 10 <= x <= 20:
    print("x är inom intervallet 10 till 20")
else:
    print("x är utanför intervallet")
# Output: x är inom intervallet 10 till 20
\end{lstlisting}

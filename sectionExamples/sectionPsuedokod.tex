\section{Kodexempel: Pseudokod}
\label{examples:psuedocode}
\exToSecRef{section:psuedocode}
\subsection*{Exempel 1: Summera tal i en lista}

\textbf{Pseudokod:}
\begin{verbatim}
START
  Lista = [10, 20, 30, 40]
  SUMMA = 0
  FOR varje TAL i Lista
    SUMMA = SUMMA + TAL
  END FOR
  Skriv SUMMA
END
\end{verbatim}

\textbf{Python-kod:}
\begin{lstlisting}[title=Summera tal i en lista]
tal_lista = [10, 20, 30, 40]
summa = 0

for tal in tal_lista:
    summa += tal

print(f"Summan är: {summa}")
# Output: Summan är: 100
\end{lstlisting}

\subsection*{Exempel 2: Kontrollera om ett tal är jämnt}

\textbf{Pseudokod:}
\begin{verbatim}
START
  Läs in TAL
  IF TAL modulo 2 = 0 THEN
    Skriv "Talet är jämnt"
  ELSE
    Skriv "Talet är udda"
  END IF
END
\end{verbatim}

\textbf{Python-kod:}
\begin{lstlisting}[title=Kontrollera om ett tal är jämnt]
tal = int(input("Ange ett tal: "))

if tal % 2 == 0:
    print("Talet är jämnt")
else:
    print("Talet är udda")
# Testa med tal som 4 och 7
\end{lstlisting}

\subsection*{Exempel 3: Hitta det största talet i en lista}

\textbf{Pseudokod:}
\begin{verbatim}
START
  Lista = [12, 45, 23, 89]
  MAX = Lista[0]
  FOR varje TAL i Lista
    IF TAL > MAX THEN
      MAX = TAL
    END IF
  END FOR
  Skriv MAX
END
\end{verbatim}

\textbf{Python-kod:}
\begin{lstlisting}[title=Hitta det största talet i en lista]
tal_lista = [12, 45, 23, 89]
största_talet = tal_lista[0]

for tal in tal_lista:
    if tal > största_talet:
        största_talet = tal

print(f"Det största talet är: {största_talet}")
# Output: Det största talet är: 89
\end{lstlisting}

\subsection*{Exempel 4: Beräkna fakultet av ett tal}

\textbf{Pseudokod:}
\begin{verbatim}
START
  Läs in TAL
  PRODUKT = 1
  FOR i = 1 TILL TAL
    PRODUKT = PRODUKT * i
  END FOR
  Skriv PRODUKT
END
\end{verbatim}

\textbf{Python-kod:}
\begin{lstlisting}[title=Beräkna fakultet av ett tal]
tal = int(input("Ange ett tal: "))
fakultet = 1

for i in range(1, tal + 1):
    fakultet *= i

print(f"Fakulteten av {tal} är: {fakultet}")
# Testa med 5 (output: 120)
\end{lstlisting}

\subsection*{Exempel 5: Fibonaccital (Iterativt)}

\textbf{Pseudokod:}
\begin{verbatim}
START
  Läs in N
  A = 0
  B = 1
  FOR i = 1 TILL N
    Skriv A
    TEMP = A + B
    A = B
    B = TEMP
  END FOR
END
\end{verbatim}

\textbf{Python-kod:}
\begin{lstlisting}[title=Fibonaccital - Iterativ metod]
n = int(input("Hur många Fibonaccital vill du visa? "))
a, b = 0, 1

for _ in range(n):
    print(a, end=" ")
    a, b = b, a + b
# Testa med 7 (output: 0 1 1 2 3 5 8)
\end{lstlisting}

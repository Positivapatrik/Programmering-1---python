\section{Kodexempel: Matematik i Python}
\label{examples:math}
\exToSecRef{section:math}
\subsection*{\texttt{math}-modulen}

\begin{lstlisting}[title=Exempel 1: Importera math-modulen]
# För att använda funktioner från math-modulen måste vi importera den
import math

# Exempel: Beräkna kvadratroten av ett tal
tal = 25
resultat = math.sqrt(tal)
print("Kvadratroten av", tal, "är", resultat)
\end{lstlisting}

\begin{lstlisting}[title=Exempel 2: Exponentiering med \texttt{math.pow()}]
import math

# math.pow() beräknar bas upphöjt till exponent
bas = 2
exponent = 3
resultat = math.pow(bas, exponent)
print(bas, "upphöjt till", exponent, "är", resultat)
\end{lstlisting}

\begin{lstlisting}[title=Exempel 3: Heltalsavrundning med \texttt{math.floor()} och \texttt{math.ceil()}]
import math

# math.floor() rundar ner till närmaste heltal
print(math.floor(5.7))  # 5

# math.ceil() rundar upp till närmaste heltal
print(math.ceil(5.7))   # 6
\end{lstlisting}

\begin{lstlisting}[title=Exempel 4: Använda matematiska konstanter som \texttt{math.pi}]
import math

# math.pi är ett konstant värde för pi
radie = 5
omkrets = 2 * math.pi * radie
print("Omkretsen av en cirkel med radie", radie, "är", omkrets)
\end{lstlisting}

\subsection*{Specialoperationer}

\begin{lstlisting}[title=Exempel 5: Exponentiering med \texttt{**}]
# Operatorn ** används för att beräkna potenser
bas = 3
exponent = 4
resultat = bas ** exponent
print(bas, "upphöjt till", exponent, "är", resultat)  # 81
\end{lstlisting}

\begin{lstlisting}[title=Exempel 6: Modulusoperatorn \texttt{\%}]
# Modulus operatorn \% beräknar resten av en division
a = 10
b = 3
rest = a % b
print("Resten av", a, "delat med", b, "är", rest)  # 1
\end{lstlisting}

\begin{lstlisting}[title=Exempel 7: Heltalsdivision med \texttt{//}]
# Heltalsdivision returnerar endast heltalsdelen av resultatet
a = 10
b = 3
kvot = a // b
print(a, "heltalsdividerat med", b, "är", kvot)  # 3
\end{lstlisting}

\subsection*{Kombinerade exempel}

\begin{lstlisting}[title=Exempel 8: Beräkna hypotenusan i en triangel]
import math

# Pythagoras sats: c = sqrt(a^2 + b^2)
a = 3
b = 4
c = math.sqrt(a**2 + b**2)
print("Hypotenusan för en triangel med sidorna", a, "och", b, "är", c)  # 5.0
\end{lstlisting}

\begin{lstlisting}[title=Exempel 9: Kontrollera om ett tal är jämnt eller udda]
# Modulus kan användas för att kontrollera jämnhet
tal = 15
if tal % 2 == 0:
    print(tal, "är ett jämnt tal.")
else:
    print(tal, "är ett udda tal.")
\end{lstlisting}

\begin{lstlisting}[title=Exempel 10: Rundning av tal till närmaste heltal]
import math

# Kombinera floor och ceil för att göra en egen rundningsfunktion
tal = 5.5

if tal - math.floor(tal) < 0.5:
    avrundat = math.floor(tal)
else:
    avrundat = math.ceil(tal)

print("Rundning av", tal, "ger", avrundat)  # 6
\end{lstlisting}

\begin{lstlisting}[title=Exempel 11: Area av en cirkel]
import math

# Beräkna area av en cirkel
radie = 7
area = math.pi * radie**2
print("Arean av en cirkel med radie", radie, "är", area)
\end{lstlisting}

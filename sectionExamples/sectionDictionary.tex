\section{Kodexempel: Dictionaries}
\label{examples:dictionary}
\exToSecRef{section:dictionary}
\subsection*{Skapa och använda dictionaries}

\begin{lstlisting}[title=Exempel 1: Skapa en dictionary]
# Skapa en dictionary för att lagra personers ålder
person_ålder = {
    "Anna": 25,
    "Björn": 32,
    "Cecilia": 19
}

# Åtkomst via nycklar
print("Annas ålder är:", person_ålder["Anna"])
# Output: Annas ålder är: 25
\end{lstlisting}

\begin{lstlisting}[title=Exempel 2: Åtkomst med \texttt{get()}]
# Använd get() för att undvika fel om nyckeln saknas
print(person_ålder.get("Daniel", "Nyckeln finns inte"))
# Output: Nyckeln finns inte
\end{lstlisting}

\subsection*{Modifiera dictionaries}

\begin{lstlisting}[title=Exempel 3: Lägga till en ny nyckel och värde]
# Lägg till en ny person
person_ålder["Daniel"] = 28
print("Uppdaterad dictionary:", person_ålder)
# Output: {'Anna': 25, 'Björn': 32, 'Cecilia': 19, 'Daniel': 28}
\end{lstlisting}

\begin{lstlisting}[title=Exempel 4: Uppdatera ett värde]
# Ändra åldern för Anna
person_ålder["Anna"] = 26
print("Annas nya ålder är:", person_ålder["Anna"])
# Output: Annas nya ålder är: 26
\end{lstlisting}

\subsection*{Ta bort från dictionaries}

\begin{lstlisting}[title=Exempel 5: Ta bort en nyckel med \texttt{pop()}]
# Ta bort Björn från dictionaryn
borttagen_ålder = person_ålder.pop("Björn")
print("Borttagen:", borttagen_ålder)
print("Uppdaterad dictionary:", person_ålder)
# Output: 
# Borttagen: 32
# Uppdaterad dictionary: {'Anna': 26, 'Cecilia': 19, 'Daniel': 28}
\end{lstlisting}

\subsection*{Iterera över dictionaries}

\begin{lstlisting}[title=Exempel 6: Iterera över nycklar]
# Skriva ut alla namn (nycklar)
for namn in person_ålder:
    print("Namn:", namn)
# Output:
# Namn: Anna
# Namn: Cecilia
# Namn: Daniel
\end{lstlisting}

\begin{lstlisting}[title=Exempel 7: Iterera över nycklar och värden]
# Skriva ut både namn och ålder
for namn, ålder in person_ålder.items():
    print(namn, "är", ålder, "år gammal")
# Output:
# Anna är 26 år gammal
# Cecilia är 19 år gammal
# Daniel är 28 år gammal
\end{lstlisting}

\begin{lstlisting}[title=Exempel 8: Iterera över värden]
# Skriva ut bara åldrar
for ålder in person_ålder.values():
    print("Ålder:", ålder)
# Output:
# Ålder: 26
# Ålder: 19
# Ålder: 28
\end{lstlisting}

\subsection*{Praktiska användningar}

\begin{lstlisting}[title=Exempel 9: Kontrollera om en nyckel finns]
# Kontrollera om en person finns
if "Anna" in person_ålder:
    print("Anna finns i dictionaryn")
# Output: Anna finns i dictionaryn
\end{lstlisting}

\begin{lstlisting}[title=Exempel 10: Räkna förekomster med en dictionary]
# Räkna antalet förekomster av varje bokstav i en sträng
bokstav_räkning = {}
text = "programmering"

for bokstav in text:
    if bokstav in bokstav_räkning:
        bokstav_räkning[bokstav] += 1
    else:
        bokstav_räkning[bokstav] = 1

print("Bokstavsräkning:", bokstav_räkning)
# Output: Bokstavsräkning: {'p': 1, 'r': 3, 'o': 1, 'g': 2, 'a': 1, 'm': 2, 'e': 1, 'i': 1, 'n': 1}
\end{lstlisting}

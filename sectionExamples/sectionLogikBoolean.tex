\section{Kodexempel: Logiska uttryck och booleanska värden}
\label{examples:boolean}
\exToSecRef{section:boolean}
\subsection*{Datatypen \texttt{bool}}

\begin{lstlisting}[title=Exempel 1: Sanna och falska värden]
# Booleanska värden är antingen True eller False
a = True
b = False
print("a är:", a)
print("b är:", b)
\end{lstlisting}

\begin{lstlisting}[title=Exempel 2: Booleanska värden från jämförelser]
# Jämförelser returnerar ett booleskt värde
print(5 > 3)   # True
print(10 < 7)  # False
print(3 == 3)  # True
print(4 != 5)  # True
\end{lstlisting}

\subsection*{Logiska operatorer: and, or, not}

\begin{lstlisting}[title=Exempel 3: Använda \texttt{and}]
# and returnerar True om båda villkoren är sanna
print(True and True)    # True
print(True and False)   # False
print(5 > 3 and 2 < 4)  # True
print(5 > 3 and 2 > 4)  # False
\end{lstlisting}

\begin{lstlisting}[title=Exempel 4: Använda \texttt{or}]
# or returnerar True om minst ett villkor är sant
print(True or False)    # True
print(False or False)   # False
print(5 > 3 or 2 > 4)   # True
print(10 < 7 or 1 == 1) # True
\end{lstlisting}

\begin{lstlisting}[title=Exempel 5: Använda \texttt{not}]
# not inverterar ett booleskt värde
print(not True)   # False
print(not False)  # True
print(not (5 > 3)) # False
\end{lstlisting}

\subsection*{Kombination av logiska uttryck}

\begin{lstlisting}[title=Exempel 6: Kombinera and\, or och not]
# Kombinera flera logiska operatorer
x = 10
y = 5
print((x > y) and (y > 2))    # True
print((x > y) or (y < 2))     # True
print(not (x == y))           # True
\end{lstlisting}

\begin{lstlisting}[title=Exempel 7: Operatorns prioritet]
# not har högre prioritet än and/or
print(not True and False)  # False
print(not (True and False)) # True
print(True or not False)    # True
print((True or False) and not False) # True
\end{lstlisting}

\subsection*{Vanliga misstag}

\begin{lstlisting}[title=Exempel 8: Glöm inte parenteser vid komplexa uttryck]
x = 10
y = 5

# Fel: operatorprioritet ger oväntat resultat
print(not x > y and y < 2)  # Detta är False

# Rätt: använd parenteser för tydlighet
print((not (x > y)) and (y < 2))  # Detta är True
\end{lstlisting}

\begin{lstlisting}[title=Exempel 9: Jämför inte direkt med \texttt{True}/\texttt{False}]
x = (5 > 3)

# Onödigt sätt
if x == True:
    print("x är sant!")

# Bättre sätt
if x:
    print("x är sant!")
\end{lstlisting}

\subsection*{Exempel i praktisk användning}

\begin{lstlisting}[title=Exempel 10: Kontrollera tillgång till system]
# Villkor: Användaren måste vara inloggad och ha adminrättigheter
inloggad = True
admin = False

if inloggad and admin:
    print("Välkommen, admin!")
else:
    print("Åtkomst nekad.")
\end{lstlisting}

\begin{lstlisting}[title=Exempel 11: Kontrollera om ett tal är i intervall]
# Kontrollera om ett tal är mellan 10 och 20
tal = 15

if tal >= 10 and tal <= 20:
    print("Talet är inom intervallet.")
else:
    print("Talet är utanför intervallet.")
\end{lstlisting}

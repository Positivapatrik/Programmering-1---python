\section{Kodexempel: Bubble Sort}
\label{examples:bubblesort}
\exToSecRef{section:bubblesort}
\subsection*{Exempel 1: Grundläggande Bubble Sort-algoritm}

\begin{lstlisting}[title=Bubble Sort - Grundläggande implementation]
def bubble_sort(lista):
    n = len(lista)
    for i in range(n):
        for j in range(0, n - i - 1):
            if lista[j] > lista[j + 1]:
                # Byt plats om elementet är större än det nästa
                lista[j], lista[j + 1] = lista[j + 1], lista[j]

# Testa med en lista
tal_lista = [64, 34, 25, 12, 22, 11, 90]
bubble_sort(tal_lista)
print("Sorterad lista:", tal_lista)
# Output: Sorterad lista: [11, 12, 22, 25, 34, 64, 90]
\end{lstlisting}

\subsection*{Exempel 2: Visualisering av varje steg i sorteringen}

\begin{lstlisting}[title=Bubble Sort - Visualisering av stegen]
def bubble_sort_med_steg(lista):
    n = len(lista)
    for i in range(n):
        print(f"Pass {i + 1}: {lista}")  # Visa listan vid varje pass
        for j in range(0, n - i - 1):
            if lista[j] > lista[j + 1]:
                lista[j], lista[j + 1] = lista[j + 1], lista[j]

tal_lista = [5, 3, 8, 6]
bubble_sort_med_steg(tal_lista)
# Output:
# Pass 1: [5, 3, 8, 6]
# Pass 2: [3, 5, 6, 8]
# Pass 3: [3, 5, 6, 8]
# Pass 4: [3, 5, 6, 8]
\end{lstlisting}

\subsection*{Exempel 3: Optimera Bubble Sort med tidig avbrytning}

\begin{lstlisting}[title=Bubble Sort - Optimerad version]
def bubble_sort_optimerad(lista):
    n = len(lista)
    for i in range(n):
        bytt = False
        for j in range(0, n - i - 1):
            if lista[j] > lista[j + 1]:
                lista[j], lista[j + 1] = lista[j + 1], lista[j]
                bytt = True
        if not bytt:
            break  # Avbryt om listan redan är sorterad

tal_lista = [1, 2, 3, 4, 5]
bubble_sort_optimerad(tal_lista)
print("Sorterad lista:", tal_lista)
# Output: Sorterad lista: [1, 2, 3, 4, 5]
\end{lstlisting}

\subsection*{Exempel 4: Bubble Sort på strängar}

\begin{lstlisting}[title=Bubble Sort - Sortera strängar]
def bubble_sort(lista):
    n = len(lista)
    for i in range(n):
        for j in range(0, n - i - 1):
            if lista[j] > lista[j + 1]:
                lista[j], lista[j + 1] = lista[j + 1], lista[j]

sträng_lista = ["banan", "äpple", "citron", "druva"]
bubble_sort(sträng_lista)
print("Sorterad lista:", sträng_lista)
# Output: Sorterad lista: ['äpple', 'banan', 'citron', 'druva']
\end{lstlisting}

\subsection*{Exempel 5: Praktisk användning av Bubble Sort}

\begin{lstlisting}[title=Bubble Sort - Praktisk användning]
# Sortera elever baserat på deras poäng
elever = [("Anna", 85), ("Björn", 75), ("Cecilia", 90), ("David", 80)]

def bubble_sort(lista):
    n = len(lista)
    for i in range(n):
        for j in range(0, n - i - 1):
            if lista[j][1] > lista[j + 1][1]:  # Sortera efter poäng
                lista[j], lista[j + 1] = lista[j + 1], lista[j]

bubble_sort(elever)
print("Sorterade elever:", elever)
# Output: Sorterade elever: [('Björn', 75), ('David', 80), ('Anna', 85), ('Cecilia', 90)]
\end{lstlisting}

\subsection*{Tips och rekommendationer}
- Bubble Sort är enkel att förstå och implementera, men ineffektiv för stora datamängder.
- Använd optimerad version för att förbättra prestanda om listan kan vara delvis sorterad.

\section{Kodexempel: Funktioner}
\label{examples:functions}
\exToSecRef{section:functions}
\subsection*{Enkla funktioner utan parametrar och returvärde}

\begin{lstlisting}[title=Exempel 1: En enkel funktion utan parametrar]
def hälsa():
    print("Hej! Välkommen till Python-programmering.")

# Anropa funktionen
hälsa()
# Output: Hej! Välkommen till Python-programmering.
\end{lstlisting}

\subsection*{Funktioner med parametrar}

\begin{lstlisting}[title=Exempel 2: Funktion med en parameter]
def hälsa(namn):
    print("Hej,", namn, "!")

# Anropa funktionen med ett argument
hälsa("Anna")
# Output: Hej, Anna!
\end{lstlisting}

\begin{lstlisting}[title=Exempel 3: Funktion med flera parametrar]
def addera(tal1, tal2):
    print("Summan är:", tal1 + tal2)

# Anropa funktionen med två argument
addera(5, 7)
# Output: Summan är: 12
\end{lstlisting}

\subsection*{Funktioner med returvärde}

\begin{lstlisting}[title=Exempel 4: Funktion som returnerar ett värde]
def kvadrat(tal):
    return tal * tal

# Spara returvärdet i en variabel
resultat = kvadrat(4)
print("Kvadraten av 4 är:", resultat)
# Output: Kvadraten av 4 är: 16
\end{lstlisting}

\begin{lstlisting}[title=Exempel 5: Funktion med både parametrar och returvärde]
def multiplicera(tal1, tal2):
    return tal1 * tal2

# Använd returvärdet direkt
print("Produkten är:", multiplicera(3, 7))
# Output: Produkten är: 21
\end{lstlisting}

\subsection*{Anropa funktioner från andra funktioner}

\begin{lstlisting}[title=Exempel 6: Funktion som använder en annan funktion]
def kvadrat(tal):
    return tal * tal

def kvadratsumma(tal1, tal2):
    return kvadrat(tal1) + kvadrat(tal2)

print("Kvadratsumman är:", kvadratsumma(2, 3))
# Output: Kvadratsumman är: 13
\end{lstlisting}

\subsection*{Namespace och lokalitet i funktioner}

\begin{lstlisting}[title=Exempel 7: Lokala och globala variabler]
def öka_värde():
    värde = 10  # Lokal variabel
    print("Inuti funktionen:", värde)

värde = 5  # Global variabel
öka_värde()
print("Utanför funktionen:", värde)

# Output:
# Inuti funktionen: 10
# Utanför funktionen: 5
\end{lstlisting}

\begin{lstlisting}[title=Exempel 8: Problem med att använda lokala variabler utanför funktioner]
def skapa_värde():
    lokalt_värde = 42
    print("Inuti funktionen:", lokalt_värde)

skapa_värde()
# Följande rad kommer orsaka ett fel eftersom "lokalt_värde" är definierad inuti funktionen
# print("Utanför funktionen:", lokalt_värde)

# Output:
# Inuti funktionen: 42
# Fel: NameError: name 'lokalt_värde' is not defined
\end{lstlisting}

\subsection*{Praktiska exempel}

\begin{lstlisting}[title=Exempel 9: Huvudfunktion i ett program]
def huvudfunktion():
    print("Detta är huvudprogrammet")
    resultat = addera(3, 4)
    print("Resultatet är:", resultat)

def addera(tal1, tal2):
    return tal1 + tal2

# Kör huvudfunktionen
huvudfunktion()

# Output:
# Detta är huvudprogrammet
# Resultatet är: 7
\end{lstlisting}

\begin{lstlisting}[title=Exempel 10: Kontrollera om program körs direkt]
def huvudfunktion():
    print("Programmet körs direkt")

if __name__ == "__main__":
    huvudfunktion()

# Output:
# Programmet körs direkt (om koden körs direkt)
\end{lstlisting}

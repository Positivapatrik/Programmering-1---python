\section{Kodexempel: Slumptal med random}
\label{examples:random}
\exToSecRef{section:random}
\subsection*{Modulimport och \texttt{randint}}

\begin{lstlisting}[title=Exempel 1: Importera modulen \texttt{random}]
# Vi måste importera modulen random innan vi kan använda dess funktioner
import random

# Generera ett heltal mellan 1 och 10
slumptal = random.randint(1, 10)
print("Ett slumptal mellan 1 och 10:", slumptal)
\end{lstlisting}

\begin{lstlisting}[title=Exempel 2: Generera flera slumptal]
import random

# Generera tre olika slumptal
print("Första slumptalet:", random.randint(1, 10))
print("Andra slumptalet:", random.randint(1, 10))
print("Tredje slumptalet:", random.randint(1, 10))
\end{lstlisting}

\subsection*{\texttt{random.random()}}

\begin{lstlisting}[title=Exempel 3: Generera ett flyttal mellan 0 och 1]
import random

# random.random() genererar ett flyttal mellan 0 och 1
flyttal = random.random()
print("Ett slumptal mellan 0 och 1:", flyttal)
\end{lstlisting}

\begin{lstlisting}[title=Exempel 4: Använd slumptal för simulering]
import random

# Simulera ett kast med en tärning, där 0.5 är gränsen för "framgång"
slumptal = random.random()
if slumptal > 0.5:
    print("Framgång!")
else:
    print("Misslyckande.")
\end{lstlisting}

\subsection*{\texttt{random.choice()}}

\begin{lstlisting}[title=Exempel 5: Välja ett slumpmässigt element från en lista]
import random

# En lista med möjliga alternativ
alternativ = ["äpple", "banan", "körsbär", "druva"]

# Välj ett slumpmässigt element från listan
val = random.choice(alternativ)
print("Det slumpmässiga valet är:", val)
\end{lstlisting}

\begin{lstlisting}[title=Exempel 6: Simulera ett tärningskast med \texttt{random.choice()}]
import random

# Definiera en lista med möjliga tärningssidor
tärning = [1, 2, 3, 4, 5, 6]

# Välj en slumpmässig sida
kast = random.choice(tärning)
print("Resultatet av tärningskastet är:", kast)
\end{lstlisting}

\subsection*{Fler funktioner i \texttt{random}}

\begin{lstlisting}[title=Exempel 7: Slumpa om en lista med \texttt{random.shuffle()}]
import random

# En lista med tal
lista = [1, 2, 3, 4, 5]

# Blanda om listan slumpmässigt
random.shuffle(lista)
print("Den omblandade listan är:", lista)
\end{lstlisting}

\begin{lstlisting}[title=Exempel 8: Generera ett slumptal inom ett intervall med decimaler]
import random

# random.uniform() genererar ett slumptal med decimaler mellan 5 och 15
slumptal = random.uniform(5, 15)
print("Ett slumptal mellan 5 och 15:", slumptal)
\end{lstlisting}

\subsection*{Syntaxfel och vanliga misstag}

\begin{lstlisting}[title=Exempel 9: Glöm inte att importera modulen!]
# Fel: random är inte importerat
slumptal = random.randint(1, 10)
print(slumptal)
# Lösning: Lägg till "import random" högst upp i programmet
\end{lstlisting}

\begin{lstlisting}[title=Exempel 10: Fel typ av argument till \texttt{randint()}]
import random

# Fel: randint() kräver två heltal som argument
# slumptal = random.randint("1", "10")
# Lösning: Skicka in heltal, inte strängar
slumptal = random.randint(1, 10)
print(slumptal)
\end{lstlisting}

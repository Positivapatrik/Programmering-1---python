\section{Kodexempel: Input (Att ta emot data från användaren)}
\label{examples:input}
\exToSecRef{section:input}
\subsection*{Grunderna för \texttt{input()}}

\begin{lstlisting}[title=Exempel 1: Läsa in text från användaren]
name = input("Vad heter du? ")
print("Hej, " + name + "!")
# Om användaren skriver: Anna
# Output: Hej, Anna!
\end{lstlisting}

\begin{lstlisting}[title=Exempel 2: Mata in ett heltal]
age = input("Hur gammal är du? ")
print("Du är " + age + " år gammal.")
# Om användaren skriver: 25
# Output: Du är 25 år gammal.
\end{lstlisting}

\begin{lstlisting}[title=Exempel 3: Fel vid aritmetiska operationer utan typkonvertering]
age = input("Hur gammal är du? ")
print(age + 5)  # Ger ett TypeError: Kan inte addera sträng och heltal
\end{lstlisting}

\subsection*{Datatypkonvertering}

\begin{lstlisting}[title=Exempel 4: Konvertera input till ett heltal]
age = int(input("Hur gammal är du? "))
print("Om fem år är du", age + 5, "år gammal.")
# Om användaren skriver: 25
# Output: Om fem år är du 30 år gammal.
\end{lstlisting}

\begin{lstlisting}[title=Exempel 5: Konvertera input till ett flyttal]
height = float(input("Hur lång är du i meter? "))
print("Du är", height, "meter lång.")
# Om användaren skriver: 1.75
# Output: Du är 1.75 meter lång.
\end{lstlisting}

\subsection*{Kontroll av datatyper}

\begin{lstlisting}[title=Exempel 6: Kontrollera datatyp efter konvertering]
number = int(input("Ange ett heltal: "))
print("Datatypen är:", type(number))
# Om användaren skriver: 42
# Output: Datatypen är: <class 'int'>
\end{lstlisting}

\subsection*{Fler exempel}

\begin{lstlisting}[title=Exempel 7: Utföra matematiska operationer]
a = int(input("Ange det första talet: "))
b = int(input("Ange det andra talet: "))
print("Summan är:", a + b)
print("Produkten är:", a * b)
# Om användaren skriver: 3 och 5
# Output: Summan är: 8
# Output: Produkten är: 15
\end{lstlisting}

\begin{lstlisting}[title=Exempel 8: Kombinera text och beräkningar]
name = input("Vad heter du? ")
birth_year = int(input("Vilket år föddes du? "))
current_year = 2024
age = current_year - birth_year
print("Hej", name + ", du är", age, "år gammal.")
# Om användaren skriver: Anna och 2000
# Output: Hej Anna, du är 24 år gammal.
\end{lstlisting}

\subsection*{Felkänslig inmatning}

\begin{lstlisting}[title=Exempel 9: Hantera felaktig inmatning]
try:
    number = int(input("Ange ett heltal: "))
    print("Du skrev:", number)
except ValueError:
    print("Det där var inte ett heltal!")
# Om användaren skriver: hej
# Output: Det där var inte ett heltal!
\end{lstlisting}

\begin{lstlisting}[title=Exempel 10: Begära inmatning tills den är korrekt]
while True:
    try:
        number = int(input("Ange ett heltal: "))
        print("Du skrev:", number)
        break
    except ValueError:
        print("Det där var inte ett heltal. Försök igen.")
# Användaren kan försöka flera gånger tills korrekt värde anges.
\end{lstlisting}

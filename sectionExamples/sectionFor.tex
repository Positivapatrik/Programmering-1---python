\section{Kodexempel: For-loopar}
\label{examples:for}
\exToSecRef{section:for}
\subsection*{Iterera över en lista}

\begin{lstlisting}[title=Exempel 1: Iterera över en lista med strängar]
# En lista med frukter
frukt_lista = ["äpple", "banan", "körsbär"]

# Iterera över varje frukt
for frukt in frukt_lista:
    print("Frukt:", frukt)
# Output:
# Frukt: äpple
# Frukt: banan
# Frukt: körsbär
\end{lstlisting}

\begin{lstlisting}[title=Exempel 2: Iterera över en lista med tal]
# En lista med tal
nummer_lista = [1, 2, 3, 4, 5]

# Beräkna kvadraten av varje tal
for nummer in nummer_lista:
    kvadrat = nummer ** 2
    print("Kvadraten av", nummer, "är", kvadrat)
# Output:
# Kvadraten av 1 är 1
# Kvadraten av 2 är 4
# ...
\end{lstlisting}

\begin{lstlisting}[title=Exempel 3: Ändra element i en lista (indirekt)]
# En lista med startvärden
nummer_lista = [10, 20, 30]

# Skapa en ny lista med dubblerade värden
dubblerade = []
for nummer in nummer_lista:
    dubblerade.append(nummer * 2)
print("Dubblerade:", dubblerade)
# Output: [20, 40, 60]
\end{lstlisting}

\subsection*{Iterera med \texttt{range()}}

\begin{lstlisting}[title=Exempel 4: Iterera med range()]
# Skriva ut tal från 0 till 4
for i in range(5):
    print("Tal:", i)
# Output:
# Tal: 0
# Tal: 1
# ...
\end{lstlisting}

\begin{lstlisting}[title=Exempel 5: Iterera med start och stopp i range()]
# Skriva ut tal från 5 till 9
for i in range(5, 10):
    print("Tal:", i)
# Output:
# Tal: 5
# Tal: 6
# ...
\end{lstlisting}

\begin{lstlisting}[title=Exempel 6: Iterera med steg]
# Skriva ut varannan siffra
for i in range(0, 10, 2):
    print("Varannan siffra:", i)
# Output:
# Varannan siffra: 0
# Varannan siffra: 2
# ...
\end{lstlisting}

\subsection*{Använda index i en lista}

\begin{lstlisting}[title=Exempel 7: Iterera över index med range()]
# En lista med frukter
frukt_lista = ["äpple", "banan", "körsbär"]

# Iterera över index
for i in range(len(frukt_lista)):
    print(f"Index {i} innehåller: {frukt_lista[i]}")
# Output:
# Index 0 innehåller: äpple
# Index 1 innehåller: banan
# ...
\end{lstlisting}

\subsection*{Kombination av listor och \texttt{range()}}

\begin{lstlisting}[title=Exempel 8: Kombinera med flera listor]
# Två listor med samma längd
namn = ["Anna", "Björn", "Cecilia"]
ålder = [22, 34, 19]

# Skriva ut namn och ålder tillsammans
for i in range(len(namn)):
    print(namn[i], "är", ålder[i], "år gammal.")
# Output:
# Anna är 22 år gammal.
# Björn är 34 år gammal.
# ...
\end{lstlisting}

\subsection*{Något mer avancerade användningar}

\begin{lstlisting}[title=Exempel 9: Summera tal i en lista]
# Lista med tal
nummer_lista = [1, 2, 3, 4, 5]

# Beräkna summan
total = 0
for nummer in nummer_lista:
    total += nummer
print("Summan är:", total)
# Output: Summan är: 15
\end{lstlisting}

\begin{lstlisting}[title=Exempel 10: Skriva ut omvänt med negativt steg]
# Iterera från 10 till 1
for i in range(10, 0, -1):
    print("Nedräkning:", i)
# Output:
# Nedräkning: 10
# Nedräkning: 9
# ...
\end{lstlisting}
